\documentclass{beamer}
\setbeamertemplate{theorems}[numbered]
\usecolortheme{dracula}
\usepackage[utf8]{inputenc}
\usepackage[
  main=spanish
]{babel}

\usepackage{amsmath,amsthm,amsfonts,amssymb}

\newcounter{Ejercicio}
%\newcounter{Ejemplo}

%\newtheorem{Theorem}{Teorema}[section]
%\newtheorem{Lemma}[Theorem]{Lema}
%\newtheorem{Corollary}[Theorem]{Corolario}
%\newtheorem{Ejercicio}[Theorem]{Ejercicio}
%
%
\newtheorem{Ejercicio}[theorem]{Ejercicio}%[count-ejercicio]

\newtheorem{Ejemplo}{Ejemplo}

%\newtheorem{Proposition}[Theorem]{Proposici\'on}
%\newtheorem{Conjecture}[Theorem]{Conjecture}
%\newtheorem{Definition}[Theorem]{Definici\'on}
%\newtheorem{Example}[Theorem]{Ejemplo}
%\newtheorem{Observation}[Theorem]{Observation}
%\newtheorem{Remark}[Theorem]{Remark}

\def \rk{{\mbox {rk}}\,}
\def \dim{{\mbox {dim}}\,}
\def \ex{\mbox{\rm ex}}
\def\df{\buildrel \rm def \over =}
\def\ind{{\mbox {ind}}\,}
\def\Vol{\mbox{Vol}}
\def\V{\mbox{Var}}
\newcommand{\comp}{\mbox{\tiny{o}}}
\newcommand{\QED}{{\hfill$\Box$\medskip}}


\def\Z{{\bf Z}}
\def\R\re
\def\V{\bf V}
\def\W{\bf W}
\def\f{\tilde{f}_{k}}
\def \e{\varepsilon}
\def \la{\lambda}
\def \vr{\varphi}
\def \R{{\bf R}}
\def \L{{\mathcal L}}

\def \re{{\mathbb R}}
\def \Q{{\mathbb Q}}
\def \cp{{\mathbb CP}}
\def \T{{\mathbb T}}
\def \C{{\bf C}}
\def \M{{\widetilde{M}}}
\def \I{{\mathbb I}}
\def \H{{\mathbb H}}
\def \lv{\left\vert}
\def \rv{\right\vert}
\def \ov{\overline}
\def \tx{{\widehat{x}}}
\def \0{\lambda_{0}}
\def \la{\lambda}
\def \ga{\gamma}
\def \de{\delta}
\def \x{\widetilde{x}}
\def \E{\mathbb{E}}
\def \y{\widetilde{y}}
\def \A{{\mathcal A}}
\def\h{{\rm h}_{\rm top}(g)}
\def\en{{\rm h}_{\rm top}}
\def\F{{\mathcal F}}
\def\co{\colon\thinspace}

\usepackage{ragged2e}  % `\justifying` text
\usepackage{booktabs}  % Tables
\usepackage{tabularx}
\usepackage{tikz}      % Diagrams
\usetikzlibrary{calc, shapes, backgrounds}
\usepackage{amsmath, amssymb}
\usepackage{url}       % `\url`s
\usepackage{listings}  % Code listings
\usepackage{dsfont}
\usepackage{mathtools}
\usepackage{stmaryrd}
\usepackage{bbold}
\usepackage{xfrac}


\title{Introducción a la Topología Simpléctica 2023-II}
\subtitle{Clase 1: Motivación dinámica y primeras definiciones} %% that will be typeset on the
\author{Pablo Suárez Serrato}
\logo{
%\includegraphics[width=2cm]{logo-IMUNAM.png}
\includegraphics[width=3cm]{LogoIMUNAM_Bco.png}
}

\begin{document}

\frenchspacing

\setbeamertemplate{caption}{\raggedright\insertcaption\par}

  \frame{\maketitle}

  \AtBeginSection[]{% Print an outline at the beginning of sections
    \begin{frame}<beamer>
      \frametitle{Contenidos}
      \tableofcontents[currentsection]
    \end{frame}}

    \section{Motivación dinámica}
%
%    \subsection{Motivación}
    \begin{frame}{Dinámica Hamiltoniana}

Al pensar en el movimiento de una part\'icula e intentar modelarlo, usamos geometr\'ia simpl\'ectica. 
 \pause

Consideremos una part\'icula, vista como un punto en $\R^{n}$, que se mueve con fuerza $\varphi =\frac{\partial U}{\partial q}$ respecto a un potencial $U$. 
 \pause
 
La ley de Newton nos dice que $\ddot{q}=-\frac{\partial U}{\partial q}$. 
 \pause

Escribimos la energ\'ia total $H=U+\frac{\dot{q}^2}{2}$ y al designar $p=\dot{q}$ llegamos a las ecuaciones de Hamilton
 
$$\frac{\partial H}{\partial p} = \dot{q} \quad , \quad \frac{\partial H}{\partial q}  = \dot{p}$$

$ (q,p)\in \R^{2n}$ son la posici\'on $(q)$ y el momento $(p)$. 

\end{frame}
    
%    
%        \begin{block}{Objetivo}
%            Comparar dos distribuciones de probabilidad considerando todas las formas de transformar, \textit{transportar} o remodelar con \alert{el menor costo} posible.
%        \end{block}
%       \pause
%       \begin{figure}
%            \centering
%            \includegraphics[scale = 0.5]{resources/opttransport1-en.png}
%            \caption{\tiny Tomado de \hyperlink{}{https://www.fv-berlin.de/en/research/research-highlight/von-a-nach-b-oder-doch-besser-nach-c-en}\par}
%        \end{figure}
    
\begin{frame}{Dinámica Hamiltoniana}

El principio de conservaci\'on de la energ\'ia implica que 

$$  \frac{d}{dt}H(q,p)=\frac{\partial H}{\partial q}\dot{q}+\frac{\partial H}{\partial p}\dot{p}=0.$$
\pause

Estas ecuaciones inducen un flujo $\varphi^{H}$. % que preserva la estructura est\'andar $\omega_0$ de $\R^{2n}$.
\pause

La evoluci\'on temporal en mec\'anica clasica la podemos pensar como un mapeo $\R \to {\rm Sp}_{2n}(\R)$, usando la derivada $d\varphi^{H}$. % del flujo $\varphi^{H}$.

\end{frame}

\begin{frame}{Dinámica Hamiltoniana}

El movimiento de $x(t)=( q(t),p(t))\in \R^{2n}$ se describe por medio del campo vectorial

$$X_{H}(q,p)= \sum\limits_{i=1}^{n}\left( \frac{\partial H}{\partial p_{i}}\cdot \frac{\partial }{\partial q_{i}}- \frac{\partial H}{\partial q_{i}}\cdot \frac{\partial }{\partial p_{i}} \right)$$

\pause

Todo campo vectorial as\'i definido (para alguna funci\'on lisa $H\co \R^{2n}\to \R$) se llama campo vectorial {\it Hamiltoniano}.

\end{frame}

\section{Primeras definiciones simplécticas}
\begin{frame}{Geometría Simpléctica}

La geometr\'ia riemanniana es la geometr\'ia de formas bilineales no-degeneradas, en cambio la geometr\'ia simpléctica es la geometr\'ia de $2$--formas bilineales, anti-sim\'etricas y no-degeneradas.
\pause


Sea ${\rm Id_{n}}$ la matriz identidad de $n\times n$ y definamos
$$\Omega:=\left (  \begin{array}{cc} 0 & {\rm Id_{n}} \\ -{\rm Id_{n}} & 0 \end{array} \right)$$ 

\pause

Al grupo de matrices que preservan $\Omega$, es decir para la que $A^{t}\Omega A =\Omega$ le llamamos ${\rm Sp}_{2n}(\R)$.


\end{frame}

\begin{frame}{Geometría Simpléctica}

 \begin{block}{Definición:}
Una variedad lisa $M^{2n}$ es {\bf simpl\'ectica} si existe un atlas con cartas cuyas derivadas son elementos de  ${\rm Sp}_{2n}(\R)$.
\end{block}


\end{frame}

\begin{frame}{Geometría Simpléctica}

Sea $\omega$ una $2$--forma de De Rham, es decir una secci\'on del haz $\bigwedge^{2}T^{\ast}M\to M$.
\pause

 \begin{block}{Definición:}
 \begin{itemize}
\item Una $2$--forma $\omega$ es cerrada si $d\omega =0$.
\pause
\vfill

 \item  En cada punto $p\in M$ la forma $\omega $ define una funci\'on $\tilde{\omega}_{p}\co T_{p}M\to T_{p}^{\ast}M\, ; \, \tilde{\omega}_{p}(v)(u)=\omega_{p}(u,v)$. 
 \vfill

 \pause
 
 \item  Decimos que $\omega$ es {\bf no-degenerada} si $\tilde{\omega}_{p}$ es un isomorfismo para todo punto $p$ de $M$.
\end{itemize}
\end{block}
\end{frame}

\begin{frame}{Geometría Simpléctica}
 \begin{block}{Definición:}
 La $2$--forma $\omega$ es {\bf simpl\'ectica} si es cerrada y no-degenerada. 
 
 Una variedad $M$ es simpl\'ectica si admite una forma simpl\'ectica. 
\end{block}
\vfill

\pause


M\'as adelante, veremos que las dos definiciones de hecho son equivalentes. 

Ser\'a consecuencia del teorema de Darboux.


\end{frame}

\section{Ejemplos y Ejercicios}
\begin{frame}{Ejemplos y Ejercicios}

\begin{Ejemplo}
Sean $(x_1, \cdots, x_{n}, y_1, \cdots , y_{n})$ coordenadas en $\R^{2n}$, la forma  

$$\omega_0=\sum\limits_{i=1}^{n}dx_{i}\wedge dy_{i}$$

es simpl\'ectica.
\end{Ejemplo}
\pause

 Es llamada la forma simpl\'ectica \emph{est\'andar}

\end{frame}

\begin{frame}{Geometría Simpléctica y Dinámica Hamiltoniana}
El flujo Hamiltoniano $\varphi^{H}$ preserva la estructura est\'andar $\omega_0$ de $\R^{2n}$.
\vfill

\pause

\begin{Ejercicio}\label{contr-anula}
 Demostrar que $X_{H}$ se caracteriza como el \'unico campo vectorial sobre $\R^{2n}$ tal que $\omega_0(X_{H}, \cdot)=dH$.
\end{Ejercicio}

\end{frame}


\begin{frame}{Ejemplos y Ejercicios}


\begin{Ejemplo}
Sean $(z_1, \cdots , z_{n})$ coordenadas en $\C^{n}$, la forma 
$$\omega_1=\frac{i}{2} \sum\limits_{k=1}^{n}dz_{k}\wedge d\bar{z}_{k}$$

es simpl\'ectica.

\vfill
       
\pause

 De hecho es la misma que $\omega_0$ al hacer el cambio de coordenadas $z_{k}=x_{k}+iy_{k}$.
\end{Ejemplo}


\end{frame}


\begin{frame}{Ejemplos y Ejercicios}

\begin{Ejercicio}
Demuestre que toda forma de area sobre una superficie ($2$--variedad) lisa, compacta y orientable es una forma simpl\'ectica. 
\end{Ejercicio}

\end{frame}

%%%

\begin{frame}{Ejemplos y Ejercicios}

\begin{Ejercicio}
Demuestre que el producto de dos variedades simpl\'ecticas es simpl\'ectica. 
\end{Ejercicio}

\end{frame}

%%%

\begin{frame}{Ejemplos y Ejercicios}

\begin{Ejercicio}
 Demostrar que $$\begin{array}{cc}                             
\underbrace{\omega_0\wedge \cdots \wedge \omega_0} & =: \omega_0^{n}
 \\ n-{\it veces} &
\end{array}$$ es una forma de volumen.
\end{Ejercicio}

\vfill
        \pause
%        
La forma $\omega_0^{n}$ se conoce como la forma de volumen de Liouville.
\end{frame}

%%%

\begin{frame}{Resumen}

\begin{itemize}
\item Dinámica y ecuaciones de Hamilton
\pause \vfill
\item Definiciones de forma y variedad simpléctica 
\pause \vfill
\item Ejemplos, formas estándar en ${\mathbf R}^{2n}$ y ${\mathbf C}^{n}$

\end{itemize}

\end{frame}




%
%
%\begin{frame}{Ejemplos y Ejercicios}
%
%
%\end{frame}


%\begin{frame}{Ejemplos y Ejercicios}
%
%
%\end{frame}
%


%
%\begin{frame}{Geometría Simpléctica}
%
%
%\end{frame}
%
%
%\begin{frame}{Dinámica Hamiltoniana}
%
%
%\end{frame}
%


\end{document}

%    \subsection{Histogramas y medidas}
%    
%    \begin{frame}{Casos discretos}
%    
%        \begin{itemize}
%            \item 
%            \pause
%            \item 
%                \begin{equation*}
%                    \alpha = \sum_{i=1}^n a_i\delta_{x_i}
%                \end{equation*}
%            donde $a_i>0$.
%    \end{itemize}
%    \end{frame}
%    
%    \begin{frame}{Caso general}
%        %        
%        \vfill
%        \pause
%        
%       
%    \end{frame}
%    
 
