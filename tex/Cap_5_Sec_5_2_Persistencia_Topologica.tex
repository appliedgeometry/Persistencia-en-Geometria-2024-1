\documentclass{beamer}
\setbeamertemplate{theorems}[numbered]
\usecolortheme{dracula}
\usepackage[utf8]{inputenc}
\usepackage[
  main=spanish
]{babel}

\usepackage{amsmath,amsthm,amsfonts,amssymb}

\newcounter{Ejercicio}
%\newcounter{Ejemplo}

%\newtheorem{Theorem}{Teorema}[section]
%\newtheorem{Lemma}[Theorem]{Lema}
%\newtheorem{Corollary}[Theorem]{Corolario}
%\newtheorem{Ejercicio}[Theorem]{Ejercicio}
%
%
\newtheorem{Ejercicio}[theorem]{Ejercicio}%[count-ejercicio]

\newtheorem{Ejemplo}{Ejemplo}

%\newtheorem{Proposition}[Theorem]{Proposici\'on}
%\newtheorem{Conjecture}[Theorem]{Conjecture}
%\newtheorem{Definition}[Theorem]{Definici\'on}
%\newtheorem{Example}[Theorem]{Ejemplo}
%\newtheorem{Observation}[Theorem]{Observation}
%\newtheorem{Remark}[Theorem]{Remark}

\def \rk{{\mbox {rk}}\,}
\def \dim{{\mbox {dim}}\,}
\def \ex{\mbox{\rm ex}}
\def\df{\buildrel \rm def \over =}
\def\ind{{\mbox {ind}}\,}
\def\Vol{\mbox{Vol}}
\def\V{\mbox{Var}}
\newcommand{\comp}{\mbox{\tiny{o}}}
\newcommand{\QED}{{\hfill$\Box$\medskip}}


\def\Z{{\bf Z}}
\def\R\re
\def\V{\bf V}
\def\W{\bf W}
\def\f{\tilde{f}_{k}}
\def \e{\varepsilon}
\def \la{\lambda}
\def \vr{\varphi}
\def \R{{\bf R}}
\def \L{{\mathcal L}}

\def \re{{\mathbb R}}
\def \Q{{\mathbb Q}}
\def \cp{{\mathbb CP}}
\def \T{{\mathbb T}}
\def \C{{\bf C}}
\def \M{{\widetilde{M}}}
\def \I{{\mathbb I}}
\def \H{{\mathbb H}}
\def \lv{\left\vert}
\def \rv{\right\vert}
\def \ov{\overline}
\def \tx{{\widehat{x}}}
\def \0{\lambda_{0}}
\def \la{\lambda}
\def \ga{\gamma}
\def \de{\delta}
\def \x{\widetilde{x}}
\def \E{\mathbb{E}}
\def \y{\widetilde{y}}
\def \A{{\mathcal A}}
\def\h{{\rm h}_{\rm top}(g)}
\def\en{{\rm h}_{\rm top}}
\def\F{{\mathcal F}}
\def\co{\colon\thinspace}

\usepackage{ragged2e}  % `\justifying` text
\usepackage{booktabs}  % Tables
\usepackage{tabularx}
\usepackage{tikz}      % Diagrams
\usetikzlibrary{calc, shapes, backgrounds}
\usepackage{amsmath, amssymb}
\usepackage{url}       % `\url`s
\usepackage{listings}  % Code listings
\usepackage{dsfont}
\usepackage{mathtools}
\usepackage{stmaryrd}
\usepackage{bbold}
\usepackage{xfrac}


\title{Seminario de Persistencia en Geometría 2024-1}
\subtitle{5.2. Complejos de \v{C}ech y Rips, y Análisis Topológico de Datos.} %% that will be typeset on the
\author{Miguel Evangelista}
\logo{
%\includegraphics[width=2cm]{logo-IMUNAM.png}
\includegraphics[width=2cm]{LogoIMUNAM_Bco.png}
}

\begin{document}

\frenchspacing

\setbeamertemplate{caption}{\raggedright\insertcaption\par}

  \frame{\maketitle}

  \AtBeginSection[]{% Print an outline at the beginning of sections
    \begin{frame}<beamer>
      \frametitle{Contenidos}
      \tableofcontents[currentsection]
    \end{frame}
}

\section{Considereaciones}


\begin{frame}{Consideracionnes}
    Sea $M$ una variedad Riemmaniana y sea $X\subseteq M$ una muestra finita de puntos de $M$.
    \newline
    \pause
    
    Dado lo anterior, surgen las siguientes preguntas.
    \begin{itemize}
        \item Teniendo sólo $X$, ¿es posible reconstruir $M$?, 
        \item ¿Qué tan bien podemos reconstruir $M$?
    \end{itemize} 
\end{frame}

\begin{frame}{Consideracionnes}
    Denotemos por $d$ la distancia de Riemann en $M$ (y por consecuencia la distancia inducida en $X$). Decimos que el espacio métrico $(X,d)$ modela una nube de datos.
    \newline
    \pause

    Para responder la pregunta anterior, se tiene que que reconstruir la topología de $M$ utilizando la geometría “local” del conjunto $X$
    \newline
    \pause
    
    Dado que uno de los principios del análisis de datos topológicos es "no confiar en grandes distancias". \textit{(Ver Michael W. Mahoney, et al, Algorithmic and statistical challenges in modern large-scale data analysis are the focus of MMDS 2008, 2008)}
\end{frame}

\section{Complejo de Rips}

\begin{frame}{Complejo de Rips}
    Consideramos un espacio métrico finito $(X, d)$. 

    Se define el complejo simplicial $R_{\alpha}(X)$ de la siguiente manera: 
    \begin{itemize}
        \item los vértices de $R_{\alpha}(X)$ son los puntos del conjunto $X$, y  
        \pause
        \item los $k+1$ puntos en X determinan un $k-$ simplejo $\sigma = [x_{0},...,x_{k}]$ si $d(x_{i}, x_{j})< \alpha \qquad \forall i, j$
    \end{itemize}
    para $0 < \alpha\in \mathbf{R}$  
\end{frame}

\begin{frame}{Módulo de Rips}
    \begin{block}{Obs.}
        \begin{itemize}
            \item El complejo de Rips está completamente determinado por su $1-$esqueleto, es de hecho, un complejo de bandera.

            \pause
            \item Para $0<\alpha\leq min_{x, y\in X, x\neq y}d(x, y)$ el complejo $R_{\alpha}(X)$ es una colección finita de puntos.
        \end{itemize}
        
        %, mientras que para $\alpha > diam(X)$, $R_{\alpha}(X)$ es un complejo de dimensión  $|X|-1$. 
    \end{block}
\end{frame}

\begin{frame}{Complejo de Rips}
    Esta construcción se ilustra en la siguiente figura 
    
    \includegraphics[width=10cm]{rips_white}
    
\end{frame}

\begin{frame}{Módulo de Rips}
    Para $\alpha \neq \beta$ existe una función  simplicial \textit{natural}
    $$i_{\alpha, \beta}: R_{\alpha}(X) \to R_{\beta}(X).$$ 
    \pause
    
    Tomando $V_{\alpha}(X) = H_{\ast}(R_{\alpha}(X))$ y $\pi_{\alpha,\beta} = (i_{\alpha,\beta})_{\ast}$, obtenemos un módulo de persistencia, que denominamos módulo de Rips.
    \pause
    
    \begin{block}{Obs.}
        Los complejos de Rips fueron introducidos por primera vez por Vietoris.
    \end{block}

\end{frame}

\section{Complejo de \v{C}ech}
\begin{frame}{Complejo de \v{C}ech}
    Para $t > 0$, escribimos $B_{t}(x)$ que denota  la bola abierta con centro en $x$ de radio $t$ con respecto a la métrica $d$.
    \newline
    \pause

    \begin{block}{Definición 5.2.1}
        Sea $U = \{U_{i}\}$ una colección finita de subconjuntos de un conjunto $A$. Definimos el complejo \v{C}ech \v{C}$(U)$ asociado a $U$ como:
        \begin{itemize}
            \item Los vértices son los conjuntos $U_i$.
            \item Una colección ordenada $\sigma = [U_{0}, \ldots, U_{k}]$  es un $k-$simplejo si $\bigcap U_{j} \neq \emptyset$.
            \item Los operadores de frontera se definen de la manera estándar, lo que permite a uno considerar la homología correspondiente $H_{*}$\v{C}$(U)$.
        \end{itemize}
    \end{block}
\end{frame}

\begin{frame}{Complejo de \v{C}ech}
    \begin{block}{Remark 5.2.2}
        Un caso especial que será de particular interés para nosotros es el siguiente: Sea $(X,d)$ y fije $t > 0$. 
        \newline
        \pause
        
        Considere la colección de bolas abiertas de radio $\frac{t}{2}$ centradas en el punto $x_{i} \in X$ denotado por $U_i = B_{\frac{t}{2}}(x_i)$. 
        \newline
        \pause
        
        Examinaremos la homología asociada al complejo \v{C}ech de esta colección $U_t = \{U_i\}$, y la denotaremos por \v{C}$_{t}(X)$\footnote{Esto es para que la notación sea similar a la del complejo Rips}.
    \end{block}
\end{frame}


\begin{frame}{Complejo de \v{C}ech}
    Variando $t>0$, de manera similar al caso del complejo Rips, podemos considerar el módulo de persistencia $H_{*}$\v{C}$(U)$ con los mapeos de persistencia inducidos por los mapeos de inclusión $i_{s,t}:$\v{C}$_{s}(X) \to $\v{C}$_{t}(X)$
    \newline
    \pause

    Notemos que los los módulos de persistencia provenientes de los complejos \v{C}ech y Rips correspondientes a un espacio métrico finito $(X, d)$ son $1-$entrelazados (after passing to a \textit{logarithmic scale}).
\end{frame}

\begin{frame}{Complejos de \v{C}ech y Rips}
    \begin{block}{Lema 5.2.3}
        Sea $(X, d)$ un espacio métrico finito. Tome  $V_{a} = H_{*}(R_{2^{a}}(X))$ y $W_{a}=H_{*}($\v{C}$_{2^{a}}(X))$, con morfismos inducidos a partir de los complejos Rips y \v{C}ech respectivamente. Entonces $V$ y $W$ son $1-$entrelazados.
    \end{block}
\end{frame}

\begin{frame}{Complejos de \v{C}ech y Rips}
    \begin{block}{Dem.}
        \begin{itemize}
            \item Si $[y_0,...,y_k]$ es un simplejo en $R_{t}(X)$ se tiene que $d(y_i,y_j) < t$, entonces $y_i \in B_{t}(y_j)$ para todo $i,j$.\\
            
            En particular, $y_0$ es un punto común para todo $B_{t}(y_{j})$, por lo que $[y_0,...,y_k]$ determina un $k-$simplejo en \v{C}$_{2t}$. Por tanto, $R_t \subset \v{C}$_{2t}$.
            \newline
            \pause
            
            \item Si $[y_0,...,y_k]$ es un simplejo \v{C}$_{t}(X)$, es decir, $\bigcap_{j} B_{\frac{t}{2}}(y_j) \neq \emptyset$, en particular para cada $y_i, y_j$ las bolas $B_{\frac{t}{2}}(yi)$ y $B_{\frac{t}{2}}(y_j)$ se intersectan, dado que $d(y_i, y_j) < t$.\\
            
            Entonces $[y_0,...,y_k]$ es un simplejo en $R_t(X) \subseteq R_{\frac{t}{2}}(X)$. Por tanto, Por tanto, \v{C}$_{2t} \subset R_t$.
        \end{itemize}
    \end{block}
\end{frame}

\begin{frame}{Complejos de \v{C}ech y Rips}
    \begin{block}{Dem.}
        \begin{itemize}
            Pasando a la “escala logarítmica” y considerando los módulos de persistencia como antes, vemos que $V$ y $W$ son $1-$entrelazados, con mapas entrelazados inducidos por la identidad.
            $\hfill\square$
        \end{itemize}
    \end{block}
\end{frame}

\begin{frame}{}
    \begin{block}{Remark 5.2.4}
       Mencionemos que los “módulos de persistencia” $V$ y $W$ reescalados logarítmicamente discutidos aquí no cumplen completamente con la definición 1.1.1 de módulos de persistencia. 
       \newline
       \pause
       
       Es decir, la propiedad $4$, que hace que los módulos de persistencia desaparezcan por la izquierda a partir de algún momento, no se cumple. Sin embargo, $V$ y $W$ son módulos de persistencia de tipo localmente finito, como se define en la sección $2.4.$
    \end{block}
\end{frame}

\begin{frame}{Ejmeplo}
    \begin{block}{Ejemplo 5.2.5}
       Considere un hexágono regular con una longitud de lado 1 (ver la siguiente figura).
       \newline
       Calculemos los códigos de barras correspondientes a sus complejos Rips y \v{C}ech.
    \end{block}

    \newline
    \pause
    \includegraphics[width=5cm]{image_14}
\end{frame}

\begin{frame}{Ejemplo}
    \begin{block}{Complejo de Rips}
       Tenemos los siguientes grupos de homología a medida que t varía.
       \newline
       \pause

       \begin{itemize}
           \item Para $0<t \leq 1$, tenemos seis puntos distintos, es decir, $H_{0}=\mathbf{R}^{6}$.
           \pause
           \item Para $1<t \leq \sqrt{3}$, tenemos un $\mathbf{S}^{1}$, entonces $H_{0}=\mathbf{R}$ y $H_{1}=\mathbf{R}$.
           \pause
           \item Para $\sqrt{3} < t \leq 2$, tenemos una esfera $\mathbf{S}^{2}$, que se obtiene pegando dos discos a lo largo de su frontera. Estos discos están creados por los triángulos que se muestran en la siguiente figura.\\

           Por tanto, $H_0=\mathbf{R}$, $H_1=0$ y $H_2=\mathbf{R}$
       \end{itemize}
    \end{block}
\end{frame}

\begin{frame}{Ejemplo}
    \begin{block}{Complejo de Rips}
       \begin{itemize}
           \item Para $t > 2$, obtenemos un $5-$simplejo creado por los vértices del hexágono, por lo que solo nos queda $H_0 = \mathbf{R}$.
       \end{itemize}
    \end{block}
    \newline
    \pause
    \includegraphics[width=11cm]{image_15}
\end{frame}


\begin{frame}{Ejemplo}
    Por último, el código de barras queda de la siguiente manera.
    \newline
    \pause
    \includegraphics[width=10cm]{image_16}
\end{frame}

\begin{frame}{Ejemplo}
    \begin{block}{Complejo de \v{C}ech}
       Tenemos los siguientes grupos de homología a medida que t varía.
       \newline
       \pause

       \begin{itemize}
           \item Para $0< t \leq 1$, tenemos, $H_{0}=\mathbf{R}^{6}$.
           \pause
           \item Para $1<t \leq \sqrt{3}$, tenemos un $\mathbf{S}^{1}$, entonces $H_{0}=\mathbf{R}$ y $H_{1}=\mathbf{R}$. Ver la siguiente figura.
       \end{itemize}
    \end{block}
    \pause
    \includegraphics[width=7cm]{image_17}
\end{frame}

\begin{frame}{Ejemplo}
    \begin{block}{Complejo de \v{C}ech}
       \begin{itemize}
           \item Para $\sqrt{3} < t \leq 2$, tenemos un $\mathbf{S}^{1}$, que se obtiene de la unión de triángulos que es homotípico a $\mathbf{S}^{1}$, es decir, todavía tenemos
           $H_{0}=\mathbf{R}$ y $H_{1}=\mathbf{R}$. 
           Ver la siguiente figura.\\
           
           \pause
           Hay que tener en cuenta que los grandes triángulos equiláteros todavía no están presentes.
       \end{itemize}
    \end{block}
    \pause
    \includegraphics[width=9cm]{image_18}
\end{frame}

\begin{frame}{Ejemplo}
    \begin{block}{Complejo de \v{C}ech}
       \begin{itemize}
           \item Para $t > 2$, obtenemos un $5-$simplejo, por lo que solo nos queda $H_0=\mathbf{R}$\\
       \end{itemize}
    \end{block}
    Por último, el código de barras queda de la siguiente manera.
    \newline
    \pause
    \includegraphics[width=10cm]{image_19}
\end{frame}

\begin{frame}{Ejemplo}
    Comparando los dos códigos de barras, notamos que el complejo Rips captura una celda bidimensional “redundante” (en el sentido de la topología del hexágono), mientras que el complejo \v{C}ech no. 
    \newline
    \pause

    \newline
    Sin embargo, observemos que el complejo \v{C}ech tiene la desventaja de ser más difícil de calcular y manejar (ver [76], Capítulo 5), ya que necesitamos conocer (y almacenar) la información sobre todos los posibles símplices (es decir, intersecciones) de cualquier número de bolas alrededor de los puntos muestreados).\\ 
\end{frame}

\begin{frame}{Ejemplo}
    Al mismo tiempo, para el complejo Rips, como mencionamos en el Ejemplo 1.1.5, la información requerida es solo sobre 1-símplejos, es decir, sobre las distancias entre cada par de puntos del conjunto de muestra  
\end{frame}

\begin{frame}{Ejemplo}
    \begin{block}{Ejercicio 5.2.6}
       Verifique que los dos códigos de barras que encontramos coincidan después de pasar a la \textit{escala logarítmica} descrita en el Lema 5.2.3.
    \end{block}
\end{frame}

\end{document}
 
