\documentclass{beamer}
\setbeamertemplate{theorems}[numbered]
\usecolortheme{dracula}
\usepackage[utf8]{inputenc}
\usepackage[
  main=spanish
]{babel}

\usepackage{amsmath,amsthm,amsfonts,amssymb}
\usepackage{bm}
\usepackage{graphicx}

\newcounter{Ejercicio}
%\newcounter{Ejemplo}

%\newtheorem{Theorem}{Teorema}[section]
%\newtheorem{Lemma}[Theorem]{Lema}
%\newtheorem{Corollary}[Theorem]{Corolario}
%\newtheorem{Ejercicio}[Theorem]{Ejercicio}
%
%
\newtheorem{Ejercicio}[theorem]{Ejercicio}%[count-ejercicio]

\newtheorem{Ejemplo}{Ejemplo}

%\newtheorem{Proposition}[Theorem]{Proposici\'on}
%\newtheorem{Conjecture}[Theorem]{Conjecture}
%\newtheorem{Definition}[Theorem]{Definici\'on}
%\newtheorem{Example}[Theorem]{Ejemplo}
%\newtheorem{Observation}[Theorem]{Observation}
%\newtheorem{Remark}[Theorem]{Remark}

\def \rk{{\mbox {rk}}\,}
\def \dim{{\mbox {dim}}\,}
\def \ex{\mbox{\rm ex}}
\def\df{\buildrel \rm def \over =}
\def\ind{{\mbox {ind}}\,}
\def\Vol{\mbox{Vol}}
\def\V{\mbox{Var}}
\newcommand{\comp}{\mbox{\tiny{o}}}
\newcommand{\QED}{{\hfill$\Box$\medskip}}
\newcommand\norm[1]{\lVert#1\rVert}
\newcommand{\calB}{{\mathcal{B}}}
\newcommand{\calC}{{\mathcal{C}}}
\DeclareMathOperator{\coim}{coim}
\DeclareMathOperator{\im}{im}
\newcommand{\restr}{\big|}  % restricted to %


\def\dint{\operatorname{d}_{\operatorname{int}}}
\def\dbot{\operatorname{d}_{\operatorname{bot}}}
\def\Z{{\bf Z}}
\def\R\re
\def\V{\bf V}
\def\W{\bf W}
\def\f{\tilde{f}_{k}}
\def \e{\varepsilon}
\def \la{\lambda}
\def \vr{\varphi}
\def \R{{\bf R}}
\def \L{{\mathcal L}}

\def \FF{{\mathbb F}}
\def \re{{\mathbb R}}
\def \Q{{\mathbb Q}}
\def \cp{{\mathbb CP}}
\def \T{{\mathbb T}}
\def \C{{\bf C}}
\def \M{{\widetilde{M}}}
\def \I{{\mathbb I}}
\def \H{{\mathbb H}}
\def \lv{\left\vert}
\def \rv{\right\vert}
\def \ov{\overline}
\def \tx{{\widehat{x}}}
\def \0{\lambda_{0}}
\def \la{\lambda}
\def \ga{\gamma}
\def \de{\delta}
\def \x{\widetilde{x}}
\def \E{\mathbb{E}}
\def \y{\widetilde{y}}
\def \A{{\mathcal A}}
\def\h{{\rm h}_{\rm top}(g)}
\def\en{{\rm h}_{\rm top}}
\def\F{{\mathcal F}}
\def\co{\colon\thinspace}

\usepackage{ragged2e}  % `\justifying` text
\usepackage{booktabs}  % Tables
\usepackage{tabularx}
\usepackage{tikz}      % Diagrams
\usetikzlibrary{calc, shapes, backgrounds}
\usepackage{amsmath, amssymb}
\usepackage{url}       % `\url`s
\usepackage{listings}  % Code listings
\usepackage{dsfont}
\usepackage{mathtools}
\usepackage{stmaryrd}
\usepackage{bbold}
\usepackage{xfrac}
\usepackage{tikz-cd}


\title{Seminario Persistencia en Geometría 2024-I}
\subtitle{Secciones 4-4.1.4} 
\author{Miguel Ángel Maurin García de la Vega}
\logo{
\includegraphics[width=1cm]{LogoIMUNAM_Bco.png}
}

\begin{document}

\frenchspacing

\setbeamertemplate{caption}{\raggedright\insertcaption\par}

  \frame{\maketitle}

  \AtBeginSection[]{% Print an outline at the beginning of sections
    \begin{frame}<beamer>
      \frametitle{Contenidos}
      \tableofcontents[currentsection]
    \end{frame}}

    \section{Objetivos Capítulo 4}

    \begin{frame}{¿Qué podemos leer de un Código de Barras?}
Este capítulo introduce funcionales Lipschitz en el espacio de códigos de barras. \pause Dichos funcionales producen invariantes en módulos de persistencia tipo finito y representaciones de grupos finitos en módulos de persistencia. \pause Se ilustran dichos invariantes en el contexto de aproximación de funciones y geometría de espacios métricos. 

\end{frame}

\section{Barras Infinitas}
    
\begin{frame}{\(\dbot\) para barras infinitas}

Recordemos que si \(\mathcal{B}\) y \(\mathcal{C}\) son códigos de barras, \(\dbot(\mathcal{B},\mathcal{C})\) se define como el ínfimo de las \(\delta\) tales que \(\mathcal{B}\) y \(\mathcal{C}\) están \(\delta\)-emparejados.
\pause
Además, un $\delta$-emparejamiento, $\mu$ debe cumplir:
\pause
	\begin{enumerate}
		\item
			${\calB}_{2\delta} \subset \coim \mu$,\pause
		\item
			${\calC}_{2\delta} \subset \im \mu$ ,\pause
		\item
			Si $\mu (I) = J$, entonces $I \subset J^{-\delta},\ J \subset I^{-\delta}$.
	\end{enumerate}
Consideremos un código de barras \(\mathcal{B}\) que consiste de \(N\) barras infinitas: 
\pause
\[(b_1, +\infty),\ (b_2, +\infty),\ \ldots,\ (b_N, +\infty),\ b_1\leq b_2 \leq \ldots \leq b_N \;.\]
Si \(\mathcal{C}\) es otro código de barras que consiste solo de barras infinitas, \pause de existir un $\delta$-emparejamiento, todas las barras de de \(\calB\) y \(\calC\) al ser infinitas son mayores que $2\delta$. \pause Por lo que lo todas las barras deben emparejarse. Tenemos así, el siguiente resultado: 
\end{frame}


\begin{frame}{\(\dbot\) para barras infinitas}
$\dbot(\calB,\calC)<+\infty \iff \calB$ y $\calC$ contienen el mismo número de barras infinitas.\\
\pause
Notemos que esto es análogo al siguiente resultado obtenido para módulos de persistencia y la distancia de $\delta$-entrelazamiento: \\
\pause
$\dint(V,W)<+\infty \iff \dim V_\infty = \dim W_\infty$ \\
\pause
(ver Ej. 1.3.2 (1) y Ej. 2.2.4)
\end{frame}

\begin{frame}{\(\dbot\) para barras infinitas}
    Consideremos entonces \(\calB\) y \(\calC\) códigos de barras consistentes de $N$ barras infinitas:
    \pause
    \[(b_1, +\infty),\ (b_2, +\infty),\ \ldots,\ (b_N, +\infty),\ b_1\leq b_2 \leq \ldots \leq b_N \;.\]
    \pause
    \[(c_1, +\infty),\ (c_2, +\infty),\ \ldots,\ (c_N, +\infty),\ c_1\leq c_2 \leq \ldots \leq c_N \;.\]
    \pause
    ¿Existe una cota inferior para $\dbot(\calB,\calC)$? Buscaremos un valor de $\delta$ que garantice un $\delta$-emparejamiento.
\end{frame}


\begin{frame}{Cota inferior para \(\dbot\)}

Recordemos que el $\delta$-emparejamiento debe cumplir $\mu (I) = J \implies I \subset J^{-\delta},\ J \subset I^{-\delta}$.
\pause
Entonces, para $N=1$ tenemos:
 \begin{center}
\includegraphics[scale=0.35]{n1_black.png}
\[\delta = |b_1 - c_1|\]
\end{center}
\pause 
Si $N=2$ tenemos que hacer más comparaciones:
\[|b_1 - c_1|,\ |b_2 - c_2|\]
\[|b_1 - c_2|,\ |b_2 - c_1|\]
\end{frame}

\begin{frame}{Cota inferior para \(\dbot\)}

Para tener un $\delta$-emparejamiento basta tomar:
\[\min \left\{ \max \left\{ |b_1 - c_1|,\ |b_2 - c_2| \right\}, \max \left\{ |b_1 - c_2|,\ |b_2 - c_1| \right\} \right\}
\]
\pause
Usando el lenguaje de permutaciones, notamos que en el primer argumento del $\min$ estamos aplicando la permutación identidad $\mathds{1}$ a las $c_i$, \pause en el segundo estamos aplicando la permutacion $(12)$.\\
\pause
Así, para $\calB, \calC$ con $N$ barras debe ser que:
\[\dbot(\calB, \calC) \geq \min_{\sigma\in S_N} \max_{i} |b_i - c_{\sigma(i)}|\]
donde $\sigma$ recorre todas las posibles permutaciones de $N$ elementos. 
\pause
El siguiente lema afirma que no es necesario considerar las permutaciones. 

 \end{frame}

\section{Lema del Emparejamiento}
 
	\begin{frame}{Lema del Emparejamiento}
	    
	\begin{block}{Lema 4.1.1}
	Para cualesquera dos conjuntos de puntos en $\R$, $b_1 \leq b_2 \leq \ldots \leq b_N$ y $c_1 \leq c_2 \leq \ldots \leq~c_N$ tenemos:
 \pause
 \[\min_{\sigma\in S_N} \max_i |b_i - c_{\sigma(i)}| = \max_i |b_i - c_i|\]
	
\end{block}
\end{frame}

\begin{frame}{Corolario Cota para \(\dbot\)}
\begin{block}{Corolario 4.1.2}
		Sean $V$ y $W$ dos módulos de persistencia con códigos de barras $\calB = \calB (V), \calC = \calB (W)$, cada uno consistente de $N$ barras infinitas. Denotamos por $b_i \text{ y } c_i $ los extremos de las barras infinitas en $\calB$ y $\calC$ (ordenados como en el Lema 4.1.1). Entonces, tenemos la siguiente cota inferior en la distancia cuello de botella entre los dos códigos de barras:
  \[d_{bot} ( \calB, \calC ) \geq \max_i | b_i - c_i |\]
\end{block}
\pause
Demostraremos ahora el Lema del Emparejamiento.
\end{frame}
\section{Demostración Lema del Emparejamiento}
\begin{frame}{Modificaciones Elementales}
	Sea
	
		\[
			\sigma =
			\begin{pmatrix}
			1 & \ldots & N \\
			\sigma(1) & \ldots & \sigma(N)
			\end{pmatrix} \;
		\]
	una permutación y supongamos que $\sigma (i+1) < \sigma (i)$.
 \pause
	Podemos modificar $\sigma$ en la permutación
		\[
		\begin{pmatrix}
		1		&\ldots&	 i & 			i+1 & 	\ldots &	N \\
		\sigma(1) & \ldots &\sigma(i+1)& \sigma(i)& \ldots & \sigma(N)
		\end{pmatrix} \;.
		\]
  \pause
	transponiendo $\sigma(i)$ y $\sigma(i+1)$. Esto es llamado una \emph{modificación elemental}.
\end{frame}
\begin{frame}{Ejercicio 4.1.3}

Mediante una secuencia de modificaciones elementales podemos transformar cualquier $\sigma \in S_N$ en la permutación identidad $\mathds{1}$.\\
\pause
Para esto, hay que identificar $\sigma(i)=1$ y transponerlo con elementos a su izquierda hasta colocarlo en su lugar, es decir, la posición 1. 
\pause
Estas transposiciones son posibles ya que se cumple $\sigma (i)= 1 < \sigma (j)$ para toda $j<i$.
\pause
Del mismo modo, repetimos el procedimiento para $\sigma(i)=2$ y así sucesivamente. \\
\pause
Ahora, para $\sigma \in S_N$ definimos:
\[T(\sigma) = \max_i |b_i - c_{\sigma(i)}|\]
\pause
Queremos mostrar que la identidad $\mathds{1}$ es el mínimo de $T(\sigma)$.
\pause
Para esto probaremos el caso $N=2$

 \end{frame}


\begin{frame}{Ejercicio 4.1.4}
Si $N=2$ tenemos: $b_1<b_2$ y $c_1<c_2$. Además, $S_N = \{\mathds{1}, (12)\}$.
\pause
Hay tres posibles configuraciones para los puntos:
\pause
\begin{enumerate}

		\item
			$b_1<b_2<c_1<c_2$,\pause
		\item
			$b_1<c_1<b_2<c_2$ ,\pause
		\item
			$b_1<c_1<c_2<b_2$.
	\end{enumerate}

\pause
En cada caso, queremos comparar 
\[\max \left\{ |b_1 - c_1|,\ |b_2 - c_2| \right\}\]
contra 
\[\max \left\{ |b_1 - c_2|,\ |b_2 - c_1| \right\}\]

\end{frame}

\begin{frame}{Ejercicio 4.1.4}
En los casos (1) y (2) se cumple que 
\[\max \left\{ |b_1 - c_1|,\ |b_2 - c_2| \right\}<\max \left\{ |b_1 - c_2|,\ |b_2 - c_1| \right\}=|b_1-c_2|\]
\pause
Gráficamente:
\begin{center}
\includegraphics[scale=0.35]{n2_black.png}
\end{center}
\end{frame}

\begin{frame}{Ejercicio 4.1.4}
    Para el caso (3) si $\max \left\{ |b_1 - c_2|,\ |b_2 - c_1| \right\}=|b_1-c_2|$, entonces $|b_1-c_1|<|b_1-c_2|$. \pause
    no puede ser que $|b_2 - c_2| > |b_1-c_2|$ ya que en tal caso $|b_2 - c_1| > |b_2-c_2|$ contradiciendo la maximalidad de $|b_1-c_2|$. \pause El caso donde \(\max =|b_2 - c_1|\) es análogo. \pause

    Por lo tanto, \(T(\mathds{1})\) es el mínimo. 
\end{frame}

\begin{frame}{Generalización a N barras}

Finalmente, generalizamos para cualquier $N$.\pause 
Sea $\sigma \in S_N$ una permutacion y $\sigma'$ una modificacion elemental de $\sigma$, que cambia $\sigma(i)$ con $\sigma(i+1)$. \pause Notamos que estamos suponiendo $\sigma(i+1)<\sigma(i)$. Veamos que $T(\sigma') \leq T(\sigma)$
\pause
Podemos aislar los índices $i, i+1$ en las definiciones de $T(\sigma)$ y $T(\sigma')$

\end{frame}

\begin{frame}{Generalización a N barras}

\[T(\sigma) = \max_j \big( |b_j - c_{\sigma(j)}| \big) \]
\[=\max \Big( \max_{j\neq i, i+1} \big( |b_j - c_{\sigma(j)} \big),\ |b_i - c_{\sigma(i)}|,\ |b_{i+1} - c_{\sigma(i+1)}| \Big) \;\]

\[T(\sigma') = \max_j \big( |b_j - c_{\sigma'(j)}| \big)\]
\[\max \Big( \max_{j\neq i, i+1} \big( |b_j - c_{\sigma(j)} \big),\ |b_i - c_{\sigma(i+1)}|,\ |b_{i+1} - c_{\sigma(i)}| \Big)\]

\end{frame}

\begin{frame}{Generalización a N barras}

Denotamos por:
\begin{itemize}
    \item $A= \max_{j\neq i, i+1} \big( |b_j - c_{\sigma(j)} \big)$
    \pause
    \item $B(\sigma) = \max \big( |b_i - c_{\sigma(i)}|,\ |b_{i+1} - c_{\sigma(i+1)}| \big)$
    \pause
    \item $B(\sigma') = \max \big( |b_i - c_{\sigma(i+1)}|,\ |b_{i+1} - c_{\sigma(i)}| \big)$
\end{itemize}
    \pause
    Por el ejercicio 4.1.4 tenemos \(B(\sigma') \leq B(\sigma)\) ya que \(i<i+1\) y \(\sigma(i+1)<\sigma(i) \). \\
    \pause Hay dos casos:
    \begin{itemize}
		\item
			$T(\sigma) = A$, \pause 
            entonces $B(\sigma) \leq A$ y dado que $B(\sigma') \leq B(\sigma)$ obtenemos $T(\sigma') = A$. \pause
		\item
			$T(\sigma) = B(\sigma)$, \pause
            entonces $A \leq B(\sigma)$ y dado que $B(\sigma') \leq B(\sigma)$ obtenemos $T(\sigma') = \max \big( A, B(\sigma') \big) \leq B(\sigma) = T(\sigma)$.
	\end{itemize}
\end{frame}
\begin{frame}{Generalización a N barras}
    Por lo tanto, si $\sigma$ es una permutación con $T(\sigma)$ mínima, 
    \pause hemos visto que toda modificación elemental da una nueva permutación $\sigma'$ con $T(\sigma')\leq T(\sigma)$. \pause
    Además, toda permutación puede transformarse en la identidad mediante un número finito de modificaciones elementales, de manera que $T(\mathds{1}) \leq T(\sigma)$. \pause
    Y así, la identidad da el mínimo:
    \[T(\sigma) = T(\mathds{1}) = \max_i |b_i - c_i|\]
    
\end{frame}

\end{document}
