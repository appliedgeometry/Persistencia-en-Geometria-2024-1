\documentclass{beamer}
\setbeamertemplate{theorems}[numbered]
\usecolortheme{dracula}
\usepackage[utf8]{inputenc}
\usepackage[
  main=spanish
]{babel}

\usepackage{amsmath,amsthm,amsfonts,amssymb}

\newcounter{Ejercicio}
%\newcounter{Ejemplo}

%\newtheorem{Theorem}{Teorema}[section]
%\newtheorem{Lemma}[Theorem]{Lema}
%\newtheorem{Corollary}[Theorem]{Corolario}
%\newtheorem{Ejercicio}[Theorem]{Ejercicio}
%
%
\newtheorem{Ejercicio}[theorem]{Ejercicio}%[count-ejercicio]

\newtheorem{Ejemplo}{Ejemplo}

%\newtheorem{Proposition}[Theorem]{Proposici\'on}
%\newtheorem{Conjecture}[Theorem]{Conjecture}
%\newtheorem{Definition}[Theorem]{Definici\'on}
%\newtheorem{Example}[Theorem]{Ejemplo}
%\newtheorem{Observation}[Theorem]{Observation}
%\newtheorem{Remark}[Theorem]{Remark}

%% Definiciones del libro:

\DeclareMathOperator{\Int}{Int}
\DeclareMathOperator{\rk}{rank} \DeclareMathOperator{\tr}{tr}
\DeclareMathOperator{\supp}{supp} \DeclareMathOperator{\spn}{span}
\DeclareMathOperator{\cl}{cl}
\DeclareMathOperator{\dv}{div} \DeclareMathOperator{\ind}{ind}
\DeclareMathOperator{\dist}{dist} \DeclareMathOperator{\res}{res}
\DeclareMathOperator{\codim}{codim} \DeclareMathOperator{\lk}{lk}
\DeclareMathOperator{\intr}{int} \DeclareMathOperator{\ord}{ord}
\DeclareMathOperator{\Hom}{Hom} \DeclareMathOperator{\Mor}{Mor}
\DeclareMathOperator{\End}{End} \DeclareMathOperator{\Aut}{Aut}
\DeclareMathOperator{\Ind}{Ind} \DeclareMathOperator{\im}{im}
\DeclareMathOperator{\Var}{Var} \DeclareMathOperator{\Vol}{Vol}
\DeclareMathOperator{\grph}{graph}
\DeclareMathOperator{\Crit}{Crit}
\DeclareMathOperator{\length}{length}
\DeclareMathOperator{\Sp}{Sp} \DeclareMathOperator{\Cal}{Cal}
\DeclareMathOperator{\rot}{rot} \DeclareMathOperator{\PSL}{PSL}
\renewcommand{\sp}{\mathfrak{sp}} \DeclareMathOperator{\Area}{Area}
\DeclareMathOperator{\CovDim}{CovDim} \DeclareMathOperator{\diam}{diam}
\DeclareMathOperator{\closure}{Closure}\DeclareMathOperator{\pb}{pb}
\DeclareMathOperator{\CZ}{CZ} \DeclareMathOperator{\GW}{GW}
\DeclareMathOperator{\Iamge}{Image}
\DeclareMathOperator{\proj}{proj}
\DeclareMathOperator{\spec}{spec}
\DeclareMathOperator{\Spec}{Spec}
\DeclareMathOperator{\dis}{dis}
\DeclareMathOperator{\Totaldim}{Totaldim}
\DeclareMathOperator{\coim}{coim}
\newcommand{\tit}{\textit}
\newcommand{\trm}{\textrm}
\newcommand{\tbf}{\textbf}
\newcommand{\trmk}[1]{\textcolor{red}{#1}}  
\newcommand{\mbf}{\mathbf}
\newcommand{\mbb}{\mathbb}
\newcommand{\mbbm}{\mathbbm}
\newcommand{\mcal}{\mathcal}
\newcommand{\mrm}{\mathrm}
\newcommand{\msc}{\mathscr}
\newcommand{\mitl}{\mathit}
\newcommand{\mfrak}{\mathfrak}
\renewcommand{\thefootnote}{\fnsymbol{footnote}}
\newcommand{\calA}{{\mathcal{A}}}
\newcommand{\calB}{{\mathcal{B}}}
\newcommand{\calC}{{\mathcal{C}}}
\newcommand{\calD}{{\mathcal{D}}}
\newcommand{\calE}{{\mathcal{E}}}
\newcommand{\calF}{{\mathcal{F}}}
\newcommand{\calG}{{\mathcal{G}}}
\newcommand{\calH}{{\mathcal{H}}}
\newcommand{\calI}{{\mathcal{I}}}
\newcommand{\calJ}{{\mathcal{J}}}
\newcommand{\calK}{{\mathcal{K}}}
\newcommand{\calL}{{\mathcal{L}}}
\newcommand{\calM}{{\mathcal{M}}}
\newcommand{\calN}{{\mathcal{N}}}
\newcommand{\calP}{{\mathcal{P}}}
\newcommand{\calQ}{{\mathcal{Q}}}
\newcommand{\calR}{{\mathcal{R}}}
\newcommand{\calS}{{\mathcal{S}}}
\newcommand{\calT}{{\mathcal{T}}}
\newcommand{\calU}{{\mathcal{U}}}
\newcommand{\calV}{{\mathcal{V}}}
\newcommand{\calW}{{\mathcal{W}}}
\newcommand{\al}{\alpha}
\newcommand{\be}{\beta}
\newcommand{\ga}{\gamma}
\newcommand{\Ga}{\Gamma}
\newcommand{\del}{\delta}
\newcommand{\Del}{\Delta}
\renewcommand{\th}{\theta}
\newcommand{\eps}{\varepsilon}
\newcommand{\epsi}{\varepsilon}
\newcommand{\et}{\eta}
\newcommand{\ka}{\kappa}
\newcommand{\la}{\lambda}
\newcommand{\La}{\Lambda}
\newcommand{\ro}{\rho}
\newcommand{\sig}{\sigma}
\newcommand{\Sig}{\Sigma}
\newcommand{\si}{\sigma}
\newcommand{\Si}{\Sigma}
\newcommand{\ph}{\varphi}
\newcommand{\vphi}{\varphi}
\newcommand{\om}{\omega}
\newcommand{\Om}{\Omega}
\newcommand{\na}{\nabla}
\newcommand{\ze}{\zeta}
\newcommand{\tPsi}{{\widetilde{\Psi}}}
\newcommand{\tilPhi}{\til{\Phi}}
\newcommand{\T}{\mathbb T}
\newcommand{\bH}{\mathbb{H} }
\newcommand{\eset}{\emptyset}
\newcommand{\sub}{\subset}
\newcommand{\setm}{\setminus}
\newcommand{\nin}{\notin}
\newcommand{\bcup}{\bigcup}
\newcommand{\bcap}{\bigcap}
\newcommand{\union}[2]{\overset{#2}{\underset{#1}{\bigcup}}}
\newcommand{\inter}[2]{\overset{#2}{\underset{#1}{\bigcap}}}
\newcommand{\id}{\mathbbm 1}               
\newcommand{\rest}[2]{#1\bigr\vert_{#2}}   
\newcommand{\Lin}{\mathcal L}
\newcommand{\ip}[1]{\langle {#1}\rangle}     
\newcommand{\ten}{\otimes}              
\newcommand{\pa}{\partial}
\newcommand{\de}{\partial}
\newcommand{\pd}[2]{\frac{\pa #1}{\pa #2}} 
\renewcommand{\d}{d}
\newcommand{\dx}{\d x}
\newcommand{\dt}{\d t}
\newcommand{\du}{\d u}
\newcommand{\ds}{\d s}
\newcommand{\ddt}{\frac{\d}{ \dt} }
\newcommand{\ddtat}[1]{\left.\ddt \right|_{t=#1}}
\newcommand{\at}[1]{\biggl. \biggr|_{#1} } % Evaluate at
%\newcommand{\conv}{\longrightarrow}        % Limit %
\newcommand{\xconv}{\xrightarrow}
\newcommand{\convas}[1]{\xrightarrow[#1]{} }
\newcommand{\nconv}{\nrightarrow}
\newcommand{\const}{\equiv}
\newcommand{\nconst}{\not \equiv}
\newcommand{\we}{\wedge}
\newcommand{\rv}{\mathrm{v}} % vector field

\newcommand{\restr}{\big|}  % restricted to %

% Limits %
\newcommand{\limn}{\displaystyle \lim_{n \to \infty}}
\newcommand{\limk}{\displaystyle \lim_{k \to \infty}}
\newcommand{\convn}{\xrightarrow[n \to \infty]{}}
\newcommand{\convk}{\xrightarrow[k \to \infty]{}}
\newcommand{\convi}{\xrightarrow[i \to \infty]{}}
\newcommand{\limx}[1]{\displaystyle\lim_{x \to #1}}
\newcommand{\limt}[1]{\displaystyle \lim_{t \to #1}}
\newcommand{\convx}[1]{\displaystyle \xrightarrow[x \to #1]{}}
\newcommand{\convt}[1]{\displaystyle \xrightarrow[t \to #1]{}}

% Convergence in norms %
\newcommand{\Lpto}{\displaystyle \xrightarrow[L_p]{}}
\newcommand{\Czto}{\displaystyle \xrightarrow[C_0]{}}


% Symplectic stuff %
\newcommand{\pois}[1]{\{#1\}} % Poisson brackets %
\DeclareMathOperator{\sgrad}{sgrad}
%\newcommand{\Ham}{\mrm{Ham}}
\newcommand{\Symp}{\mrm{Symp}}
\newcommand{\tCal}{\widetilde{\Cal}}
\newcommand{\tHam}{\widetilde{\Ham}}


\newcommand{\poisnm}[1]{\|\{#1\}\|} % Poisson bracket norm (not specified norm), not really in use.. %


% Persistence stuff %
\DeclareMathOperator{\Rips}{Rips}


% Misc %
\newcommand{\ol}{\overline}
\newcommand{\til}{\tilde}
\newcommand{\wtil}{\widetilde}
\newcommand{\ul}{\underline}
\newcommand{\imp}{\Rightarrow}
\newcommand{\limp}{\Leftarrow}
\newcommand{\disp}{\displaystyle}
\newcommand{\spc}{\,,\,}
\newcommand{\Spc}{\,,\,}
\newcommand{\into}{\hookrightarrow}
\newcommand{\trans}{\pitchfork}

\newcommand{\vect}[1]{\overrightarrow{#1}}

\DeclareMathOperator{\Ad}{Ad}
\DeclareMathOperator{\ad}{ad}
\DeclareMathOperator{\Vect}{Vect}



\def\ep{\epsilon}
\def\k{{\bf k}}
\def\d{d_{\rm Hofer}}
\def\f{{\mathfrak{f}}}
\def\g{{\mathfrak{g}}}
\def\R{\mathbb{R}}
\def\Z{\mathbb{Z}}
\def\N{\mathbb{N}}
\def\C{\mathbb{C}}
\def\Q{\mathbb{Q}}
\def\F {\mathbb{F}} 
\def\G{\mathcal{G}}
\def\K{\mathcal{K}}
\def\I{\mathbb{I}}
\def\V{\mathbb{V}}
\def\W{\mathbb{W}}
\def\S{\mathcal{S}}
\def\p{\mathfrak{p}}
\def\ev{\mathbf{ev}}
\def\CF{{\rm CF}}
\def\HF{{\rm HF}}
\def\SH{{\rm SH}}
\def\HM{{\rm HM}}
\def\Ham{{\rm Ham}}
\def\SBM{{\rm SBM}}
\def\RBM{{\rm RBM}}
\def\CBM{{\rm CBM}}
\def\sign{{\rm sign}}
\def\Diff{{\rm Diff}}
\def\Cont{{\rm Cont}}
\def\coker{{\rm coker}}
\def\Im{{\rm Im}}
\newcommand{\Osc}{{\rm Osc}}


%%%



 \def \rk{{\mbox {rk}}\,}
 \def \dim{{\mbox {dim}}\,}
 \def \ex{\mbox{\rm ex}}
 \def\df{\buildrel \rm def \over =}
 \def\ind{{\mbox {ind}}\,}
 \def\Vol{\mbox{Vol}}
 \def\V{\mbox{Var}}
 \newcommand{\comp}{\mbox{\tiny{o}}}
 \newcommand{\QED}{{\hfill$\Box$\medskip}}


% \def\Z{{\bf Z}}
% \def\R\re
% \def\V{\bf V}
% \def\W{\bf W}
% \def\f{\tilde{f}_{k}}
% \def \e{\varepsilon}
% \def \la{\lambda}
% \def \vr{\varphi}
% \def \R{{\bf R}}
% \def \L{{\mathcal L}}

% \def \re{{\mathbb R}}
% \def \Q{{\mathbb Q}}
% \def \cp{{\mathbb CP}}
% \def \T{{\mathbb T}}
% \def \C{{\bf C}}
% \def \M{{\widetilde{M}}}
% \def \I{{\mathbb I}}
% \def \H{{\mathbb H}}
% \def \lv{\left\vert}
% \def \rv{\right\vert}
% \def \ov{\overline}
% \def \tx{{\widehat{x}}}
% \def \0{\lambda_{0}}
% \def \la{\lambda}
% \def \ga{\gamma}
% \def \de{\delta}
% \def \x{\widetilde{x}}
% \def \E{\mathbb{E}}
% \def \y{\widetilde{y}}
% \def \A{{\mathcal A}}
% \def\h{{\rm h}_{\rm top}(g)}
% \def\en{{\rm h}_{\rm top}}
% \def\F{{\mathcal F}}
\def\co{\colon\thinspace}

\usepackage{ragged2e}  % `\justifying` text
\usepackage{booktabs}  % Tables
\usepackage{tabularx}
\usepackage{tikz}      % Diagrams
\usetikzlibrary{calc, shapes, backgrounds}
\usepackage{amsmath, amssymb}
\usepackage{url}       % `\url`s
\usepackage{listings}  % Code listings
\usepackage{dsfont}
\usepackage{mathtools}
\usepackage{stmaryrd}
\usepackage{bbold}
\usepackage{xfrac}


\title{Caítulo 2: Códigos de Barras}
\subtitle{Distancia del cuello de botella y Teorema de Isometría} %% that will be typeset on the
\author{Haydeé Peruyero}
\logo{
%\includegraphics[width=2cm]{logo-IMUNAM.png}
\includegraphics[width=2cm]{LOGO CCM-BLANCO.png}
}

\begin{document}

\frenchspacing

\setbeamertemplate{caption}{\raggedright\insertcaption\par}

  \frame{\maketitle}

   \AtBeginSection[]{% Print an outline at the beginning of sections
     \begin{frame}<beamer>
       \frametitle{Contenidos}
       \tableofcontents[currentsection]
     \end{frame}}

 %   \section{Motivación dinámica}
%
%    \subsection{Motivación}


\begin{frame}{Definiciones}

Dado un intervalo $I = (a,b]$, denotemos por $I^{-\delta} = (a-\delta, b+\delta]$ el intervalo obtenido de $I$ al expandirlo por $\delta$ en ambos lados. \\[0.2cm] \pause

Sea $\calB$ un código de barras. Para $\epsi > 0$, denotemos por ${\calB}_\epsi$ el conjunto de todas las barras de $\calB$ de longitud mayor que $\epsi$. Estamos forzando a no considerar las barras pequeñas. \\[0.2cm] \pause

Un {\color{green}\emph{emparejamiento (match)}} entre dos multi-conjuntos finitos $X,Y$ es una biyección $\mu : X' \to Y'$, donde $X' \subset X,\ Y' \subset Y$. En este caso, $X' = \coim \mu,\ Y' = \im \mu$, y decimos que los elementos de $X'$ y $Y'$ están {\color{green}\emph{emparejados (matched)}}. \pause 

Si un elemento aparece en un multi-conjunto varias veces, debemos tratar sus copias diferentes por separado ya que solo algunas de estas podrían estar emparejadas. 

\end{frame}


\begin{frame}{\emph{$\delta$-emparejamiento} entre códigos de barras}
    \textbf{Definición 2.2.1:} Un {\color{green}\emph{$\delta$-emparejamiento}} entre dos códigos de barras $\calB$ y $\calC$ es un emparejamiento $\mu : {\calB}\to {\calC}$, tal que:
	\begin{enumerate}
		\item
			${\calB}_{2\delta} \subset \coim \mu$,
		\item
			${\calC}_{2\delta} \subset \im \mu$ ,
		\item
			Si $\mu (I) = J$, entonces $I \subset J^{-\delta},\ J \subset I^{-\delta}$. \\[0.4cm] \pause
	\end{enumerate}

 \textbf{Ejercicio 2.2.2} Si $\calB, \calC$ son \emph{$\delta$-emparejados} y si $\calC, \calD$ son $\gamma$-emparejados, entonces $\calB, \calD$ son $(\delta+\gamma)$-emparejados.
\end{frame}

\begin{frame}{Distancia de cuello de botella}
    \textbf{Definición 2.2.3:} La {\color{green}\emph{distancia de cuello de botella}}, $d_{bot} (\calB, \calC)$, entre dos códigos de barras $\calB, \calC$ está definida como el ínfimo sobre todas las $\delta$ para los cuales existe un $\delta$-emparejamiento entre $\calB$ y $\calC$. \\[0.4cm] \pause

    \textbf{Ejercicio 2.2.4:} Dos códigos de barras $\calB$ y $\calC$ están $\delta$-emparejados con un $\delta$ finito si y sólo si tienen el mismo número de rayos infinitos. \\[0.4cm] \pause

    \textbf{Corolario 2.2.5:} La distancia de cuello de botella $d_{bot}$ es una distancia en el espacio de códigos de barras con la misma cantidad de rayos infinitos.

\end{frame}

\begin{frame}{Ejemplo}
    Consideremos los módulos de persistencia $\F(a,b] \text{ y } \F(c,d]$ de los intervalos ($a,b,c,d\in \R$) y sus códigos de barras correspondientes $\calB =\{ (a,b] \}$ y $\calC = \{ (c,d] \}$. \\ \pause
    
	Entonces existe ya sea un $\delta$-emparejamiento vacío entre ellos para $\delta = \max \big( \frac{b-a}{2}, \frac{d-c}{2} \big)$ ( ya que entonces las longitudes de ambos intervalos no exceden $2\delta$), o un $\delta-$emparejamiento $(a,b] \to (c,d]$  para $\delta = \max (|a-c|, |b-d|)$. \\ \pause
 
 Así, $d_{bot} (\calB, \calC ) \leq \min \Big( \max \big( \frac{b-a}{2}, \frac{d-c}{2} \big), \max (|a-c|, |b-d|) \Big)$.
\end{frame}

\begin{frame}{Teorema de Isometría}

    \textbf{{\color{green}Teorema 2.2.8:}} La función $V \mapsto \calB (V)$ es una isometría, i.e.\ para cualquiera dos módulos de persistencia $V, W$, tenemos que:
	$$
		d_{int} (V,W) = d_{bot} (\calB (V), \calB(W)) \.
	$$

\pause 

\textbf{Nota:} En el caso de que ambos $\calB (V)$ y $\calB (W)$ no tengan el mismo número de barras infinitas, entonces ambas distancias $d_{int} (V,W)$ y $d_{bot} (\calB, \calC )$ son infinitas por definición.\\[0.2cm] \pause

\textbf{Ejercicio 2.2.10:} Probar que para cualesquiera dos códigos de barras $\calB$ y  $\calC$ tenemos que $d_{bot} (\calB, \calC) = 0$ si y sólo si $\calB = \calC$.\\
Deducir que $d_{int} (V,W) = 0$ si y sólo si $V = W$.

\end{frame}

\end{document}

