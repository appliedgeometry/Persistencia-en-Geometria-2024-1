
\documentclass{beamer}
\setbeamertemplate{theorems}[numbered]
\usecolortheme{dracula}
\usepackage[utf8]{inputenc}
\usepackage[
  main=spanish
]{babel}

\usepackage{amsmath,amsthm,amsfonts,amssymb}
%\usefonttheme[onlymath]{serif}

\newcounter{Ejercicio}
%\newcounter{Ejemplo}

%\newtheorem{Theorem}{Teorema}[section]
%\newtheorem{Lemma}[Theorem]{Lema}
%\newtheorem{Corollary}[Theorem]{Corolario}
%\newtheorem{Ejercicio}[Theorem]{Ejercicio}
%
%
\newtheorem{Ejercicio}[theorem]{Ejercicio}%[count-ejercicio]

\newtheorem{Ejemplo}{Ejemplo}

%\newtheorem{Proposition}[Theorem]{Proposici\'on}
%\newtheorem{Conjecture}[Theorem]{Conjecture}
%\newtheorem{Definition}[Theorem]{Definici\'on}
%\newtheorem{Example}[Theorem]{Ejemplo}
%\newtheorem{Observation}[Theorem]{Observation}
%\newtheorem{Remark}[Theorem]{Remark}
\def\matching{apareamiento}
\def\matched{apareados}

\def \rk{{\mbox {rk}}\,}
\def \dim{{\mbox {dim}}\,}
\def \ex{\mbox{\rm ex}}
\def\df{\buildrel \rm def \over =}
\def\ind{{\mbox {ind}}\,}
\def\Vol{\mbox{Vol}}
\def\V{\mbox{Var}}
\newcommand{\comp}{\mbox{\tiny{o}}}
\newcommand{\QED}{{\hfill$\Box$\medskip}}


\def\Z{{\bf Z}}
\def\R\re
\def\V{\bf V}
\def\W{\bf W}
\def\f{\tilde{f}_{k}}
\def \e{\varepsilon}
\def \la{\lambda}
\def \vr{\varphi}
\def \R{{\bf R}}
\def \L{{\mathcal L}}

\def \re{{\mathbb R}}
\def \Q{{\mathbb Q}}
\def \cp{{\mathbb CP}}
\def \T{{\mathbb T}}
\def \C{{\bf C}}
\def \M{{\widetilde{M}}}
\def \I{{\mathbb I}}
\def \H{{\mathbb H}}
\def \lv{\left\vert}
\def \rv{\right\vert}
\def \ov{\overline}
\def \tx{{\widehat{x}}}
\def \0{\lambda_{0}}
\def \la{\lambda}
\def \ga{\gamma}
\def \de{\delta}
\def \x{\widetilde{x}}
\def \E{\mathbb{E}}
\def \y{\widetilde{y}}
\def \A{{\mathcal A}}
\def\h{{\rm h}_{\rm top}(g)}
\def\en{{\rm h}_{\rm top}}
\def\F{{\mathcal F}}
\def\co{\colon\thinspace}

\usepackage{ragged2e}  % `\justifying` text
\usepackage{booktabs}  % Tables
\usepackage{tabularx}
\usepackage{tikz}      % Diagrams
\usetikzlibrary{calc, shapes, backgrounds}
\usepackage{amsmath, amssymb}
\usepackage{url}       % `\url`s
\usepackage{listings}  % Code listings
\usepackage{dsfont}
\usepackage{mathtools}
\usepackage{stmaryrd}
\usepackage{bbold}
\usepackage{xfrac}


\title{Parte I: Cap\'itulo 3}
\subtitle{3.4 Demostraci\'on de los Lemas 3.3.1 y 3.3.2 \scalebox{0.6}{\emph{(Proofs of Lemma 3.3.1 and Lemma 3.3.2)}}} 
\author{Eduardo Vel\'azquez}
\logo{
%\includegraphics[width=2cm]{logo-IMUNAM.png}
\includegraphics[scale=0.1]{cimat-logo-w.png}
}

\begin{document}

\frenchspacing

\setbeamertemplate{caption}{\raggedright\insertcaption\par}

  \frame{\maketitle}

  %\AtBeginSection[]{% Print an outline at the beginning of sections
    %\begin{frame}<beamer>
    %  \frametitle{Contenidos}
    %  \tableofcontents[currentsection]
    %\end{frame}}

    %\section{Motivación dinámica}
%
%    \subsection{Motivación}
\begin{frame}{Recordemos...}
\begin{itemize}
\item $(V,\pi^V)$ y $(W,\pi^W)$ son m\'odulos de persistencia.\\
\vspace{1em}
\item $\delta >0$, $f:V\rightarrow W[\delta]$ y $g:W\rightarrow V[\delta]$ son morfismos de entrelazamiento, es decir, $g[\delta]\circ f =\Phi_{V}^{2\delta}$ y $f[\delta]\circ g =\Phi_{W}^{2\delta}$\\
\vspace{1em}
\item $\Phi_{V}^{2\delta}=\pi^{V}_{t,t+2\delta}$
\end{itemize}


\end{frame}


\begin{frame}{Demostraci\'on del Lema 3.3.1}
\begin{block}{Lema 3.3.1}
Sean $\left(V,\pi^V \right)$ y  $\left(W,\pi^W \right)$ dos m\'odulos de persistencia $\delta-$entrelazados, y sean 
\begin{itemize}
\item[] \hfill$f:V\rightarrow W [\delta]$ \hfill$\,$
\item[] \hfill y \hfill$\,$
\item[] \hfill $g:W\rightarrow V [\delta]$\hfill$\,$
\end{itemize}
sus morfismos de entrelazamiento. Considere el mapeo suprayectivo
$f:V\rightarrow \mbox{im}\,f$ y el \matching ~inducido $\mu_{sur}:\mathcal{B}(V)\rightarrow\mathcal{B}(\mbox{im}\,f)$. Entonces 
\begin{enumerate}
\item $\mbox{coim} \, \mu_{sur}\supseteq \mathcal{B}(V)_{2\delta}$,
\item $\mbox{im}\,\mu_{sur}=\mathcal{B}(\mbox{im}\,f)$ ,
\item $\mu_{sur}$ mapea $(b, d]\in \mbox{coim}\,\mu_{sur}$ a $(b, d^\prime ]$, donde $d^\prime \in [d - 2\delta, d]$.
\end{enumerate}
\end{block}
\end{frame}


\begin{frame}{2. Por demostrar: $\mbox{im}\,\mu_{sur}=\mathcal{B}(\mbox{im}\,f)$.}
%\begin{enumerate}\setcounter{enumi}{1}
%\item Por demostrar: $\mbox{im}\,\mu_{sur}=\mathcal{B}(\mbox{im}\,f)$.
%\end{enumerate}
\begin{proof}
Se sigue de la Proposici\'on 3.2.8, ya que $\mu_{sur}:\mathcal{B}(V)\rightarrow\mathcal{B}(\mbox{im}\,f)$ y,
\begin{center}
\scalebox{0.8}{\begin{minipage}{0.75\textwidth}
\begin{block}{Proposi\'on 3.2.8}
Si existe un mapeo suprayectivo del m\'odulo $(V,\pi)$ al m\'odulo $(W,\theta)$, entonces el mapeo inducido
$\mu:\mathcal{B}\rightarrow \mathcal{C}$ satisface:
\begin{itemize}
\item $\mbox{im}\,\mu_{sur}=\mathcal{C}$,
\item $\mu_{sur}\,(b,d]=(b,e]$ con $d\geq e\,$.
\end{itemize}
\end{block}
\end{minipage}}
\end{center}
\end{proof}
\end{frame}


\begin{frame}{1. Por demostrar: $\mbox{coim} \, \mu_{sur}(f)\supseteq \mathcal{B}(V)_{2\delta}$.}
%\begin{enumerate}\setcounter{enumi}{0}
%\item Por demostrar: $\mbox{coim} \, \mu_{sur}(f)\supseteq \mathcal{B}(V)_{2\delta}$.
%\end{enumerate}
\begin{proof}\renewcommand{\qedsymbol}{}
El mapeo suprayectivo se construye como $\mu_{sur}: \mathcal{B}(V ) \rightarrow \mathcal{B}(\mbox{im}\, f )$. Por hip\'otesis,
 \begin{center}
 \includegraphics[scale=0.24]{./diagrams/1.png}
 \end{center}
 y por la Afirmaci\'on 3.2.13,
  \begin{center}
 \includegraphics[scale=0.24]{./diagrams/2.png}.
 \end{center}
Notemos que por construcci\'on, $\mbox{coim}\,\mu_{sur}(\Phi_{V}^{2\delta})=\mathcal{B}(V)_{2\delta}$
$\,$, y $\mathcal{B}(V)_{2\delta}$ son las barras de $\mathcal{B}(V)$ de longitud al menos $2\delta$
\end{proof}
\end{frame}


\begin{frame}
Las barras de $\mathcal{B}(V)$ y las barras
\begin{gather*}
\mathcal{B}(\mbox{im}\, \Phi_{V}^{2\delta})=\{ (b, d -2\delta] :
(b, d] \in \mathcal{B}(V ), d - b > 2\delta\}
\end{gather*}
\vspace{0.5em}
 est\'an \emph{en correspondencia} (matched) usando el criterio de \lq\lq mayor longitud primero\rq\rq, pero no as\'i las barras de $\mathcal{B}(V)$ de menor longitud a $2\delta$. Esto significa que 
 \begin{gather*}
\mbox{ coim}\mu_{sur}(f)\supseteq \mbox{coim}\,\mu_{sur} (\Phi^{2\delta}_{V} )=\mathcal{B}(V)_{2\delta}\,.
 \end{gather*}
 $\,\hfill \qed$
\end{frame}


\begin{frame}{3. Por demostrar: $\mu_{sur}$ mapea $(b, d]\in \mbox{coim}\,\mu_{sur}$ a $(b, d^\prime ]$, donde $d^\prime \in [d - 2\delta, d]$.}
\begin{proof}\renewcommand{\qedsymbol}{}
Sea $(b, d] \in \mathcal{B}(V )$. Analicemos $\mu_{sur} (f )(b, d]$.
\begin{itemize}
\item Caso 1. $d - b > 2\delta$. Por la Afirmaci\'on 3.2.8, 
\begin{gather*}
(b, d] \overset{\mu_{sur}(f)}{\mapsto}  (b, d^\prime ]\mbox{, p.a. }d^\prime \leq d\,.
\end{gather*}
Y adem\'as,
\begin{gather*}
(b, d^{\prime} ] \overset{\mu_{sur}(g[\delta])}{\mapsto}  (b, d^{\prime\prime} ]\mbox{, p.a. }d^{\prime\prime} \leq d^{\prime}\,,
\end{gather*}
adem\'as, $(b, d^{\prime\prime} ] = (b, d - 2\delta]$ .
\end{itemize}
Por tanto, $d - 2\delta\leq d^{\prime}\leq d$ .
\end{proof}
\end{frame}


\begin{frame}{3. Por demostrar: $\mu_{sur}$ mapea $(b, d]\in \mbox{coim}\,\mu_{sur}$ a $(b, d^\prime ]$, donde $d^\prime \in [d - 2\delta, d]$.}
\begin{center}
\includegraphics[scale=0.2]{./diagrams/3.png}
\end{center}
\begin{itemize}
\item Caso 2. $d - b \leq 2\delta$. El intervalo $(b, d]$ (en la coimagen de $\mu_{sur} (f )$) est\'a \emph{asociado} (matched) a
$(b, d^{\prime} ]$ con $d \geq d^{\prime}$ . Pero $d^{\prime} > b \geq d - 2\delta$, por tanto, $d^{\prime} \in [d - 2\delta, d]$.\\
 $\,\hfill \qed$
\end{itemize}

\end{frame}


%%%%%%%%%%%%%%%%%%%%%%%%%%%%%%%%%

\begin{frame}{Demostraci\'on del Lema 3.3.2}
\begin{block}{Lema 3.3.2}
Sean $\left(V,\pi^V \right)$ y  $\left(W,\pi^W \right)$ dos m\'odulos de persistencia $\delta-$entrelazados, y sean 
\begin{itemize}
\item[] \hfill$f:V\rightarrow W [\delta]$ \hfill$\,$
\item[] \hfill y \hfill$\,$
\item[] \hfill $g:W\rightarrow V [\delta]$\hfill$\,$
\end{itemize}
sus morfismos de entrelazamiento. Considere el mapeo inyectivo
$\mbox{im}\,f \rightarrow W[\delta]$ y el \matching ~inducido $\mu_{inj}:\mathcal{B}(\mbox{im}\,f)\rightarrow\mathcal{B}(W[\delta])$. Entonces 
\begin{enumerate}
\item $\mbox{coim}\,\mu_{inj}=\mathcal{B}(\mbox{im}\,f)$ ,
\item $\mbox{im} \, \mu_{inj}\supseteq \mathcal{B}(W[\delta])_{2\delta}$,
\item $\mu_{inj}$ mapea $(b, d^{\prime}]\in \mbox{coim}\,\mu_{inj}$ a $(b^{\prime}, d^\prime ]$, donde $b^\prime \in [b - 2\delta, b]$.
\end{enumerate}
\end{block}
\end{frame}



\begin{frame}{1. Por demostrar: $\mbox{coim}\,\mu_{inj}=\mathcal{B}(\mbox{im}\,f)$.}
\begin{proof}
Se sigue de la Proposici\'on 3.2.8,% ya que $\mu_{sur}:\mathcal{B}(V)\rightarrow\mathcal{B}(\mbox{im}\,f)$ y,
\begin{center}
\scalebox{0.8}{\begin{minipage}{0.75\textwidth}
\begin{block}{Proposi\'on 3.2.5}
Si existe un mapeo inyectivo del m\'odulo $(V,\pi)$ al m\'odulo $(W,\theta)$, entonces el mapeo inducido
$\mu_{inj}:\mathcal{B}\rightarrow \mathcal{C}$ satisface:
\begin{itemize}
\item $\mbox{coim}\,\mu_{inj}=\mathcal{B}$,
\item $\forall\, (b,d]\in \mathcal{B}$, $(b,d]=(c,d]$ con $c\leq b\,$.
\end{itemize}
\end{block}
\end{minipage}}
\end{center}
\end{proof}
\end{frame}

\begin{frame}{2. Por demostrar: $\mbox{im} \, \mu_{inj}\supseteq \mathcal{B}(W[\delta])_{2\delta}$}
\begin{proof}\renewcommand{\qedsymbol}{}
Por hip\'otesis, $f[\delta]\circ g =\Phi_{W}^{2\delta}$, es decir,
\begin{center}
\includegraphics[scale=0.2]{./diagrams/4.png}
\end{center}
Entonces, $\mbox{im}\,\Phi_{W}^{2\delta}\subseteq\mbox{im}\,f[\delta]\subseteq W[2\delta]$. Es decir, existen mapeos inyectivos $i,\,j,\,k$, tales que
\begin{center}
\includegraphics[scale=0.2]{./diagrams/5.png}
\end{center}
\end{proof}
\end{frame}

\begin{frame}{2. Por demostrar: $\mbox{im} \, \mu_{inj}\supseteq \mathcal{B}(W[\delta])_{2\delta}$}
%\begin{proof}%\renewcommand{\qedsymbol}{}
Por la afirmaci\'on 3.2.13,
\begin{center}
\includegraphics[scale=0.2]{./diagrams/6.png}
\end{center}
Podemos ver que, 
\begin{itemize}
\item $\mbox{im}\, \mu_{inj} (k) =\mathcal{B}(W [2\delta])_{2\delta}$,
\item Vimos que $\mbox{im}\,\Phi_{W}^{2\delta}\subseteq\mbox{im}\,f[\delta]\subseteq W[2\delta]$. Notemos que
\begin{gather*}
\mathcal{B}(\mbox{im}\,\Phi_{W}^{2\delta})=\left\{ (b,d-2\delta]\,:\,(b,d\in \mathcal{B}(W),\,d-b>2\delta)\right\},\\
\\
\mathcal{B}(W[2\delta])=\left\{ (b-2\delta,d-2\delta)\,:\,(b,d]\in\mathcal{B}(W)\right\},
\end{gather*}
y $\mu_{inj} (k)(b, d - 2\delta] = (b - 2\delta, d - 2\delta]$,\\
$\,$\hfill  $\therefore \mbox{im}\, \mu_{inj} (i) \supseteq \mbox{im}\,\mu_{inj} (k) =\mathcal{B}(W [2\delta])_{2\delta}$.\hspace{3em}$\qed$
\end{itemize}

%\end{proof}

\end{frame}



\begin{frame}{3. Por demostrar: $\mu_{inj}$ mapea $(b, d^{\prime}]\in \mbox{coim}\,\mu_{inj}$ a $(b^{\prime}, d^\prime ]$, donde $b^\prime \in [b - 2\delta, b]$}
\begin{proof}
Sea $(b,d]\in\mathcal{B}(\mbox{im}\,f[\delta])$, y $\mu_{inj}(i)(b,d])=(b^{\prime},d]$ p.a. $b^{\prime}$ tal que $(b^{\prime},d]\in\mathcal{B}(W[2\delta])$. Por la proposici\'on 3.2.5, $b^{\prime}\leq b$.
\begin{itemize}
\item Caso $d-b^{\prime}\leq 2\delta$. Se tiene $b^{\prime}\geq d-2\delta > b-2\delta$.
\item Caso $d-b^{\prime}>2\delta$. Existe un intervalo $(b^{\prime} + 2\delta, d] \in \mathcal{B}(\mbox{im}\, \Phi^{2\delta}_{W})$ tal que
\begin{gather*}
\mu_{inj} (k)(b^{\prime} + 2\delta, d] = \mu_{inj} (i)(b, d] = (b^{\prime}, d],
\end{gather*}
por lo que $b \leq b^{\prime} + 2\delta$. Por tanto, $b-2\delta \leq b^{\prime} \leq b$ .
\end{itemize}
\end{proof}
\end{frame}

\end{document}
Por suposici\'on, el siguiente diagrama conmuta



