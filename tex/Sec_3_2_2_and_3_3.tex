\documentclass{beamer}
\setbeamertemplate{theorems}[numbered]
\usecolortheme{dracula}
\usepackage[utf8]{inputenc}
\usepackage[
  main=spanish
]{babel}

\usepackage{amsmath,amsthm,amsfonts,amssymb}

\newcounter{Ejercicio}
%\newcounter{Ejemplo}

%\newtheorem{Theorem}{Teorema}[section]
%\newtheorem{Lemma}[Theorem]{Lema}
%\newtheorem{Corollary}[Theorem]{Corolario}
%\newtheorem{Ejercicio}[Theorem]{Ejercicio}
%
%
\newtheorem{Ejercicio}[theorem]{Ejercicio}%[count-ejercicio]

\newtheorem{Ejemplo}{Ejemplo}

%\newtheorem{Proposition}[Theorem]{Proposici\'on}
%\newtheorem{Conjecture}[Theorem]{Conjecture}
%\newtheorem{Definition}[Theorem]{Definici\'on}
%\newtheorem{Example}[Theorem]{Ejemplo}
%\newtheorem{Observation}[Theorem]{Observation}
%\newtheorem{Remark}[Theorem]{Remark}

\def \rk{{\mbox {rk}}\,}
\def \dim{{\mbox {dim}}\,}
\def \ex{\mbox{\rm ex}}
\def\df{\buildrel \rm def \over =}
\def\ind{{\mbox {ind}}\,}
\def\Vol{\mbox{Vol}}
\def\V{\mbox{Var}}
\newcommand{\comp}{\mbox{\tiny{o}}}
\newcommand{\QED}{{\hfill$\Box$\medskip}}


\def\Z{{\bf Z}}
\def\R\re
\def\V{\bf V}
\def\W{\bf W}
\def\f{\tilde{f}_{k}}
\def \e{\varepsilon}
\def \la{\lambda}
\def \vr{\varphi}
\def \R{{\bf R}}
\def \L{{\mathcal L}}

\def \re{{\mathbb R}}
\def \Q{{\mathbb Q}}
\def \cp{{\mathbb CP}}
\def \T{{\mathbb T}}
\def \C{{\bf C}}
\def \M{{\widetilde{M}}}
\def \I{{\mathbb I}}
\def \H{{\mathbb H}}
\def \lv{\left\vert}
\def \rv{\right\vert}
\def \ov{\overline}
\def \tx{{\widehat{x}}}
\def \0{\lambda_{0}}
\def \la{\lambda}
\def \ga{\gamma}
\def \de{\delta}
\def \x{\widetilde{x}}
\def \E{\mathbb{E}}
\def \y{\widetilde{y}}
\def \A{{\mathcal A}}
\def\h{{\rm h}_{\rm top}(g)}
\def\en{{\rm h}_{\rm top}}
\def\F{{\mathcal F}}
\def\co{\colon\thinspace}

\usepackage{ragged2e}  % `\justifying` text
\usepackage{booktabs}  % Tables
\usepackage{tabularx}
\usepackage{tikz}      % Diagrams
\usetikzlibrary{calc, shapes, backgrounds}
\usepackage{amsmath, amssymb}
\usepackage{url}       % `\url`s
\usepackage{listings}  % Code listings
\usepackage{dsfont}
\usepackage{mathtools}
\usepackage{stmaryrd}
\usepackage{bbold}
\usepackage{xfrac}


\title{Seminario de Persistencia en Geometría 2024-1}
\subtitle{3.2.2. Construcción de emparejamientos inducidos. \\ 3.3. Principales lemas y demostración del teorema.} %% that will be typeset on the
\author{Miguel Evangelista}
\logo{
%\includegraphics[width=2cm]{logo-IMUNAM.png}
\includegraphics[width=2cm]{LogoIMUNAM_Bco.png}
}

\begin{document}

\frenchspacing

\setbeamertemplate{caption}{\raggedright\insertcaption\par}

  \frame{\maketitle}

  \AtBeginSection[]{% Print an outline at the beginning of sections
    \begin{frame}<beamer>
      \frametitle{Contenidos}
      \tableofcontents[currentsection]
    \end{frame}
}

\section{Construcción de emparejamientos inducidos (Continuación)}

% \section{La distancia de Gromov-Hausdorff}
%
%    \subsection{Motivación}

\begin{frame}{Correspondencia Suprayectiva}
    Consideremos la categoría de los códigos de barras con las correspondencias como morfismos.
    \newline
    \pause
    
    Anteriormente, se estableció una correspondencia entre los objetos de la categoría de módulos de persistencia y los de la categoría de códigos de barras.
    
    %Es decir, un módulo de persistencia se corresponde con su código de barras, V ( V ). 

\end{frame}

\begin{frame}{Correspondencia Suprayectiva}    
    Sean $V$ y $W$ dos modulos de persistencia.
    \newline
    \pause
    
    Dado un morfismo $f:V\to W$, se tiene que existe  una correspondencia $\mu(f)$ entre $\mathcal{B}(V)$ y $\mathcal{B}(W)$. 
    \newline 
    \pause
    
    Surge la siguiente duda: ¿El mapeo $\mu(f)$ da un functor entre las dos categorías?

\end{frame}

\subsection{Ejemplo}
\begin{frame}{Ejemplo}
    Consideremos el siguiente ejemplo:
    \begin{block}{Ejemplo}
        Sea $I$ un intervalo y consideremos los siguientes módulos de persistencia
        $$U = V = \mathbb{F}(I) \otimes \mathbb{F}(I), \textit{   } W = \mathbb{F}(I),$$

        y dos morfismos $f:U\to V$ y $g:V \to W$ dados por $f(s, t)=(s, 0)$ y $g(s, t) = t$.
    \end{block}
\end{frame}

\begin{frame}{Ejemplo}
    Notemos lo siguiente:
    \begin{itemize}
        \item $\mu(f)$ empareja una copia de $I$ con una copia de $I$ en $\mathcal{B}(V)$ y la segunda copia permanece sin emparejar.
        \item $\mu(g)$ empareja una copia de $I$ con $I$ en $\mathcal{B}(W)$
    \end{itemize}
    \pause

    Entonces, $\mu(g)\circ \mu(f)$ empareja una copia de $I$ con $I$ en $\mathcal{B}(W)$ y la segunda queda sin emparejar. 
    \newline
    \pause

    Por otro lado también se tiene que $g\circ f=0$.
\end{frame}

\begin{frame}{Ejemplo}
    Ahora notemos lo siguiente:
    \newline
    \pause 

    Si restringimos los morfismos entre módulos de persistencia de tal manera que sólo estemos en el caso inyectivo o supreyectivo.  
    \newline
    \pause
    
    El mapeo que lleva un módulo de persistencia $V$ a su código de barras $\mathcal{B}(V)$ y un morfismo $f:V\to W$ a la correspondencia inducida ($\mu_{inj}$ ó $\mu_{sur}$) es un functor, tal y como se indica en la afirmación 3.2.13.
    
\end{frame}

\subsection{Afirmación 3.2.13}
\begin{frame}{Afirmación 3.2.13}
    \begin{block}{Afirmación 3.2.13}
        Consideremos el siguiente diagrama conmutativo en la categoría de módulos de persistencia con sólo el caso inyectivo o suprayectivo. 
    \end{block}
    \pause
    
    \includegraphics[width=7cm]{image_6}
    \pause
    
    Entonces el diagrama correspondiente a nivel de códigos de barras también conmuta:
    \pause

    \includegraphics[width=7cm]{image_9}
    \pause
    
    donde $\mu_{q}$ denota $\mu_{inj}$ o $\mu_{sur}$, respectivamente.
\end{frame}


\begin{frame}{Afirmación 3.2.13}
    Demostramos la functorialidad en el caso de inyectividad
    \newline
    \pause 
    
    Para comenzar la demostración debemos recordar la definición 3.2.4 y la proposición 3.2.1. 
    \newline
    \pause
    
    \begin{block}{Dem (de la Afirmación 3.2.13)}
        Para cualquier $d \in \mathbb{R}\cup \{+\infty\}$, los códigos de barras correspondientes a $U, V, W$ consisten en las siguientes barras que terminan en $d$:
    \end{block}
\end{frame}

\begin{frame}{Afirmación 3.2.13}
    \begin{align*} 
        \mathcal{B}(U) &: (a_{1},d] \supseteq \ldots \supseteq (a_{k},d]\\
        \mathcal{B}(V) &: (b_{1},d] \supseteq \ldots \supseteq (b_{k},d] \supseteq \ldots \supseteq (b_{l},d]\\
        \mathcal{B}(W) &: (c_{1},d] \supseteq \ldots \supseteq (c_{k},d] \supseteq \ldots \supseteq (c_{l},d] \supseteq \ldots \supseteq (c_{q},d]\\
    \end{align*}
    donde $k \leq l \leq q$. 
\end{frame}

\begin{frame}{Afirmación 3.2.13}
    Además, $\mu_{inj}(f)(a_{i},d]=(b_{i},d]$, $\mu_{inj}(g)(b_{i},d]=(c_{i},d]$ y $\mu_{inj}(h)(a_{i},d]=(c_{i},d]$ para cualquier $1\leq  i \leq k$.
    \newline
    \pause

    Esto es válido para cualquier $d$, por lo que la diagrama en el nivel de los códigos de barras conmuta.
\end{frame}

\section{Principales lemas y demostración del teorema.}
\subsection{Introducción}
\begin{frame}{Introducción}
    Supongamos que $(V,\pi^{V})$ y $(W,\pi^{W})$ son $\delta-$ intercalados.
    \newline
    \pause
    
    Lo anterior quiere decir, que existen dos morfismos $f:V \to W [\delta]$ y $g : W \to V [\delta]$, tales que $g[\delta] \circ f = \Phi^{ 2\delta}_{V}$ y $f[\delta] \circ g = \Phi^{2\delta}_{W}$, donde $\Phi^{2\delta}_{V} = \pi^{V}_{t,t+2\delta}$ y $\Phi_{W}^{2\delta}=\pi^{W}_{t,t+2\delta}$. 
    \newline 
    \pause
    
    El objetivo principal es construir una $\delta$-correspondencia entre $\mathcal{B}(V)$ y $\mathcal{B}(W)$.
    \newline
    \pause

    Recordemos la notación $\mathcal{B}_{\epsilon}$ indica la colección de barras de longitud $> \epsilon$ en un código de barras $\mathcal{B}$.
    
\end{frame}

\subsection{Lema 3.3.1}
\begin{frame}{Lema 3.3.1}
    \begin{block}{Lema 3.3.1}
        Supongamos que tenemos dos módulos de persistencia $\delta-$entrelazados $(V, \pi^V)$ y $(W, \pi^W)$, es decir, existen morfismos $f:V \to W[\delta]$ y $g : W \to V[\delta]$. 
        \newline
        \pause

        Consideremos un mapeo suprayectivo $f:V\to im(f)$  y el emparejamiento inducido $\mu_{sur}:\mathcal{B}(V) \to \mathcal{B}(im (f))$\footnote{Recordemos la Definición 3.2.7}, Entonces:
        \newline
        \pause

        \begin{itemize}
            \item $coim \mu_{sur} \supseteq \mathcal{B}(V)_{2\delta}$
            \item $im \mu_{sur} = \mathcal{B}(im(f))$, y
            \item $\mu_{sur}$ lleva $(b, d] \in coim \mu_{sur}$ a $(b,d']$, \\
            con $d' \in [d - 2\delta, d]$.
        \end{itemize}
    \end{block}

\end{frame}

\subsection{Lema 3.3.2}
\begin{frame}{Lema 3.3.2}
    \begin{block}{Lema 3.3.2}
        Supongamos que tenemos dos módulos de persistencia $\delta-$entrelazados $(V, \pi^V)$ y $(W, \pi^W)$, es decir, existen morfismos $f:V \to W[\delta]$ y $g : W \to V[\delta]$. 
        \newline
        \pause

        Consideremos el apeo inyectivo $im(f)\to W[\delta]$ y el emparejamiento inducido $\mu_{inj}:\mathcal{B}(im (f)) \to \mathcal{B}(W[\delta])$\footnote{Recordemos la Definición 3.2.4}, Entonces:
        \newline
        \pause

        \begin{itemize}
            \item $coim \mu_{inj} = \mathcal{B}(im (f))$
            \item $im \mu_{inj} \supseteq \mathcal{B}(W[\delta])_{2\delta}$, y
            \item $\mu_{inj}$ lleva $(b, d'] \in coim \mu_{inj}$ a $(b',d']$, \\
            con $b' \in [b - 2\delta, b]$.
        \end{itemize}
    \end{block}

    Los lemas 3.3.1 y 3.3.2 se prueban en la sección 3.4. 
\end{frame}

\subsection{Teo 3.1.2}
\begin{frame}{Teo 3.1.2}
    \begin{block}{Teo 3.1.2}
        Sean $V$ y $W$ dos módulos de persistencia y sus respectivos códigos de barras $\mathcal{B}(V)$ y $\mathcal{B}(V)$. Entonces $$d_{int}(V,W) \geq d_{bot}(B(V),B(W)).$$ 

    \end{block}
\end{frame}

\subsection{Demo del Teo 3.1.2}
\begin{frame}{Demostración del Teo 3.1.2}
    \begin{block}{Demostración del Teo 3.1.2}
        Consideremos:
        \begin{itemize}
            \item El emparejamiento inducido $\mu(f) = \mu_{inj}\circ\mu_{sur}$, y
            \pause
            
            \item la función $\Psi_{\delta}: \mathcal{B}(W[\delta])\to \mathcal{B}(W)$ definido para todas las barras como $(a, b]\to(a+\delta,b+\delta]$.
        \end{itemize}
        \pause
        
        Notemos que $\Psi_{\delta}$ desplaza cada barra a la derecha en $\delta$ 
        \pause
        
        Afirmamos que $\Psi_{\delta}\circ\mu(f)$ es un $\delta-$entrelazamiento entre $\mathcal{B}(V)$ y $\mathcal{B}(W)$. 

    \end{block}
\end{frame}

\begin{frame}{Demostración del Teo 3.1.2}
    \begin{block}{Demostración del Teo 3.1.2}
        Utilizando el Lema 3.3.1 y Lema 3.3.2, obtenemos el siguiente diagrama: 
        \pause
        
        \includegraphics[width=9cm]{image_10}
        
        
    \end{block}
\end{frame}

\begin{frame}{Demostración del Teo 3.1.2}
    \begin{block}{Demostración del Teo 3.1.2}
        Por el Lema $3.3.2$, se tiene que una barra $(b, d] \in \mathcal{B}(V)_{2\delta}$ se lleva a $\mu_{sur}(b, d]=(b, d']\in \mathcal{B}(im (f))$, donde $d - 2\delta \leq d' \leq d$. 
        \newline
        \pause
        
        Entonces, dicha barra se lleva a $\mu_{inj} (b, d'] = (b',d'] \in \mathcal{B}(W[\delta])_{2\delta}$, donde $b - 2\delta \leq b' \leq b$, por el Lema $3.3.1$. 
        \newline 
        \pause
        
        Por último, $(b', d']$ se desplaza a la derecha en $\delta$, es decir, se lleva a $(b + \delta, d' + \delta]$ mediante $\Psi_{\delta}$
    \end{block}
\end{frame}

\begin{frame}{Demostración del Teo 3.1.2}
    \begin{block}{Demostración del Teo 3.1.2}
        En particular, se tiene que toda barra en $\mathcal{B}(V)_{2\delta}$ 
        % (es decir, una barra "suficientemente larga") 
        es de hecho emparejada por $\mu(f)$.
        \newline
        \pause
        
        De una forma análoga, se puede comprobar que toda barra en $\mathcal{B}(W)_{2\delta}$ coincide.
    \end{block}
\end{frame}


\begin{frame}{Demostración del Teo 3.1.2}
    \begin{block}{Demostración del Teo 3.1.2}
        Además, a partir de la información sobre $b'$ y $d'$ obtenemos lo siguiente:
        \newline
        \pause
        \includegraphics[width=9cm]{image_13}
        \pause
        Por lo que $\Psi_{\delta}\circ\mu(f)$ es una $\delta-$emparejamiento entre $\mathcal{B}(V)$ y $\mathcal{B}(W)$. Por lo tanto, 
        \newline 
        \pause
        $$d_{bot}(\mathcal{B}(V), \mathcal{B}(W)) \leq d_{int}(V, W ).$$
        $\hfill\square$
    \end{block}
\end{frame}
\end{document}
 
