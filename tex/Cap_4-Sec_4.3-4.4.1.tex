
\documentclass{beamer}
\setbeamertemplate{theorems}[numbered]
\usecolortheme{dracula}
\usepackage[utf8]{inputenc}
\usepackage[
  main=spanish
]{babel}

\usepackage{amsmath,amsthm,amsfonts,amssymb}
%\usefonttheme[onlymath]{serif}

\newcounter{Ejercicio}
%\newcounter{Ejemplo}

%\newtheorem{Theorem}{Teorema}[section]
%\newtheorem{Lemma}[Theorem]{Lema}
\newtheorem{claim}[theorem]{Afirmaci\'on}
\newtheorem*{claim*}{Afirmaci\'on}
%\newtheorem{Ejercicio}[Theorem]{Ejercicio}
%
%
\newtheorem{Ejercicio}[theorem]{Ejercicio}%[count-ejercicio]
%\newtheorem{Afirmacion}[theorem]{claim}
\newtheorem{Ejemplo}{Ejemplo}
\usepackage{cancel}
%\newtheorem{Proposition}[Theorem]{Proposici\'on}
%\newtheorem{Conjecture}[Theorem]{Conjecture}
%\newtheorem{Definition}[Theorem]{Definici\'on}
\newtheorem{example2}[theorem]{Ejemplo}
%\newtheorem{Observation}[Theorem]{Observation}
%\newtheorem{Remark}[Theorem]{Remark}
\def\matching{apareamiento}
\def\matched{apareados}

\def \rk{{\mbox {rk}}\,}
\def \dim{{\mbox {dim}}\,}
\def \ex{\mbox{\rm ex}}
\def\df{\buildrel \rm def \over =}
\def\ind{{\mbox {ind}}\,}
\def\Vol{\mbox{Vol}}
\def\V{\mbox{Var}}
\newcommand{\comp}{\mbox{\tiny{o}}}
\newcommand{\QED}{{\hfill$\Box$\medskip}}


\def\Z{{\bf Z}}
\def\R\re
\def\V{\bf V}
\def\W{\bf W}
\def\f{\tilde{f}_{k}}
\def \e{\varepsilon}
\def \la{\lambda}
\def \vr{\varphi}
\def \R{{\bf R}}
\def \L{{\mathcal L}}

\def \re{{\mathbb R}}
\def \Q{{\mathbb Q}}
\def \cp{{\mathbb CP}}
\def \T{{\mathbb T}}
\def \C{{\bf C}}
\def \M{{\widetilde{M}}}
\def \I{{\mathbb I}}
\def \H{{\mathbb H}}
\def \lv{\left\vert}
\def \rv{\right\vert}
\def \ov{\overline}
\def \tx{{\widehat{x}}}
\def \0{\lambda_{0}}
\def \la{\lambda}
\def \ga{\gamma}
\def \de{\delta}
\def \x{\widetilde{x}}
\def \E{\mathbb{E}}
\def \y{\widetilde{y}}
\def \A{{\mathcal A}}
\def\h{{\rm h}_{\rm top}(g)}
\def\en{{\rm h}_{\rm top}}
\def\F{{\mathcal F}}
\def\co{\colon\thinspace}

\usepackage{ragged2e}  % `\justifying` text
\usepackage{booktabs}  % Tables
\usepackage{tabularx}
\usepackage{tikz}      % Diagrams
\usetikzlibrary{calc, shapes, backgrounds}
\usepackage{amsmath, amssymb}
\usepackage{url}       % `\url`s
\usepackage{listings}  % Code listings
\usepackage{dsfont}
\usepackage{mathtools}
\usepackage{stmaryrd}
\usepackage{bbold}
\usepackage{xfrac}


\title{Parte I: Cap\'itulo 4}
\subtitle{4.3 Funci\'on de multiplicidad\\ \scalebox{0.6}{\emph{(The Multiplicity Function)}}\\ 4.4.1 Representationes en m\'odulos de persistencia:\\ Desarrollo te\'orico\\ \scalebox{0.6}{\emph{(Representations on persistence modules - Theoretical development)}}} 
\author{Eduardo Vel\'azquez}
\logo{
%\includegraphics[width=2cm]{logo-IMUNAM.png}
\includegraphics[scale=0.1]{cimat-logo-w.png}
}

\begin{document}

\frenchspacing

\setbeamertemplate{caption}{\raggedright\insertcaption\par}

  \frame{\maketitle}

  %\AtBeginSection[]{% Print an outline at the beginning of sections
    %\begin{frame}<beamer>
    %  \frametitle{Contenidos}
    %  \tableofcontents[currentsection]
    %\end{frame}}

    %\section{Motivación dinámica}
%
%    \subsection{Motivación}
\begin{frame}{4.3 Funci\'on de Multiplicidad}
Sea $\mathcal{B}$ un c\'odigo de barras, e $I\subset \mathbb{R}$ un intervalo finito.\\
\vspace{1em}
 Denotemos $m(\mathcal{B},I)$ al n\'umero de barras en $\mathcal{B}$ que contienen a $I$.\\
 \vspace{1em}
  Adem\'as, sea $I=(a,b]$ y $c\leq \frac{b-a}{2}$ denotemos $I^c=(a+c,b-c]$.
   \vspace{4em}
  \begin{block}{Definici\'on 4.3.2}
  Se define la {\bfseries Funci\'on de multiplicidad} como
    \begin{gather*}
\mu_{k}(\mathcal{B})=\mbox{sup}\{ c \,| \,\exists\, \mbox{un intervalo finito I de longitud} > 4c,\\
\,\hspace{5em} \mbox{tal que } m(\mathcal{B},I)=m(\mathcal{B},I^{2c})=k\}\,.
  \end{gather*}
  \end{block}
  
  \scalebox{0.7}{En caso de que no exista dicha $c$, $\mu_{k}(\mathcal{B})=0$ .}
\end{frame}


\begin{frame}
Dado $k\in\mathbb{N}$, $\mu_{k}(\mathcal{B})=sup(\{c\})$
\begin{itemize}
\item[] $c\in\mathbb{R}$ tal que:
\item Podemos encontrar un intervalo $I$ tal que $\frac{\mbox{Longitud}\,I}{4}>c$.
\item $I$ est\'a contenido en $\mathcal{B}$
\item $m(\mathcal{B},I)=k$
\item $m(\mathcal{B},I^{2c}=(a+2c,b-2c]\,)=k$
\end{itemize}
\begin{center}
\includegraphics[scale=0.4]{diagrams/fig06.png}
\end{center}
\end{frame}

\begin{frame}
\begin{center}
\includegraphics[scale=0.3]{./diagrams/fig07.png}
\end{center}
En palabras, dado $k\in \mathbb{N}$, la funci\'on de multiplicidad busca la m\'axima ventana, i.e. un intervalo $I$ de longitud $>4c$ en $\mathbb{R}$, tal que arriba de ella y del intervalo acortado $I^{2c}$ existan exactamente $k$ barras.
\end{frame}

\begin{frame}
\begin{minipage}{0.25\textwidth}

\end{minipage}\hfill\scalebox{0.8}{\begin{minipage}{0.75\textwidth}
\begin{block}{Ejercicio 4.3.1}
Sean $\mathcal{B}$ y $\mathcal{C}$ dos c\'odigos de barras tales que
\begin{gather*}
d_{bot}\left( \mathcal{B},\,\mathcal{C}\right)<c\,,
\end{gather*}
adem\'as, sea $I$ un intervalo de longitud $>4c$ tal que $m(\mathcal{B},I)=m(\mathcal{B},I^{2c})=m_0$. Entonces
\begin{gather*}
m(\mathcal{C}, I^c ) = m_0\, .
\end{gather*}
\end{block}
\end{minipage}}
$\,$\\
Por el ejercicio 4.3.1, podemos deducir que para dos c\'odigos de barras $\mathcal{B},\,\mathcal{C}$ y cualquier $k\in \mathbb{N}$,
\begin{gather} \tag{16}\label{eq:16}
|\mu_k(\mathcal{B})-\mu_k(\mathcal{C})|\leq d_{bot}(\mathcal{B},\mathcal{C})\,.
\end{gather}
$\,$\\
Una aplicaci\'on de~(\ref{eq:16}) es la siguiente.
\end{frame}

\begin{frame}
\begin{block}{Definici\'on 4.3.3}
\begin{itemize}
\item Decimos que un m\'odulo de persistencia $(V,\pi)$ sobre $\mathbb{R}$ admite una {\bfseries estructura compleja} $J$ si existe un isomorfismo $J:V\rightarrow V$ que satisface $J^2=-\mathbb{1}$.

\item En tal caso, llamaremos a $(V,\pi)$ {\bfseries m\'odulo de persistencia \emph{complejo}}. \\

\item Se sigue adem\'as que $dim V_t$ es par $\forall t\in\mathbb{R}$ .
\end{itemize}
\end{block}
\end{frame}

\begin{frame}
%\setcounter{section}{3}
%\setcounter{subsection}{3}
%\setcounter{theorem}{3}
\begin{claim*}{(4.3.4)} Sea $\mathcal{B}$ el c\'odigo de barra asociado a un m\'odulo de persistencia $(V,\pi)$ que admite una estructura compleja, entonces $m(\mathcal{B},I)$ es par para cualquier intervalo $I$. En particular, se sigue que para un m\'odulo de persistencia complejo $(V,\pi)$, obtenemos $\mu_k\left(\mathcal{B}(V)\right)=0$ para todo entero impar $k\in \mathbb{N}$.
\end{claim*}
\begin{proof}\let\qed\relax
Sea $I=(a,b]$ y $\tilde{a}\in\mathbb{R}$ suficientemente cerca de $a$ tal que $a<\tilde{a}<b$. 
\vspace{-3em}
\begin{center}
\includegraphics[scale=0.35]{./diagrams/fig08a.png}
\end{center}
Cualquier barra que contenga a $I$ contribuye en $+1$ a $dim(\mbox{im}\,\pi_{\tilde{a},b})$, es decir, $m(\mathcal{B},I)=dim(\mbox{im}\,\pi_{\tilde{a},b})$.
\end{proof}
\end{frame}


\begin{frame}
\begin{center}
\includegraphics[scale=0.35]{./diagrams/fig08b.png}
\end{center}
Como $\pi_{\tilde{a},b}J_{\tilde{a}}=J_{b}\pi_{\tilde{a},b}$, entonces $J_b(\mbox{im}\,\pi_{\tilde{a},b)}\subset \mbox{im}\pi_{\tilde{a},b}$, es decir, $J^\prime:=J_b\vert_{\mbox{im}\,\pi_{\tilde{a},b}}$ tambi\'en satisface $(J^\prime)^2=-\mathbb{1}\Rightarrow dim(\mbox{im}\pi_{\bar{a},b})$ es par.\vfill\hfill$\qed$
\end{frame}


\begin{frame}
Sea $\mu_{odd}(\mathcal{B}):=$  $\mbox{max}\atop {\scalebox{0.5}{j\,odd}}$ $\mu_{j}(\mathcal{B})$.
\vspace{1em}
\begin{claim*}{}Sea $(V,\pi)$ un m\'odulo de persistencia. Entonces para todo m\'odulo $(W,\theta)$ que admita una estructura compleja, la distancia de entrelazamiento entre ellos est\'a acotada por abajo:
\begin{gather*}
d_{int}\left((V,\pi),\,(W,\theta) \right)\geq \mu_{odd}\left( \mathcal{B}(V,\pi)\right)\,.
\end{gather*}
\end{claim*}
\begin{itemize}
\item[] Implicaciones:
\item La distancia de entrelazamiento entre cualquier m\'odulo de persistencia $(V,\pi)$y la colecci\'on de m\'odulos complejos est\'a acotada por abajo por $\mu_{odd}\left( \mathcal{B}(V,\pi)\right)$.
\item En el caso de que $\mu_{odd}\left( \mathcal{B}(V,\pi)\right)>0$, obtenemos una restricci\'on para aproximar un m\'odulo de persistencia $(V,\pi)$ dado por un m\'odulo de persistencia complejo.
\end{itemize}
\end{frame}


\begin{frame}
\begin{proof}
Por el Teorema de Isometr\'ia, la desigualdad~(\ref{eq:16}) y la Afirmaci\'on 4.3.4, si $(W,\theta)$ es un m\'odulo complejo, entonces
\begin{eqnarray*}
d_{int}\left((V,\pi),\,(W,\theta) \right)&=& d_{bot}\left( \mathcal{B}(V,\pi),\,\mathcal{B}(W,\theta)\right)\\
&\geq& |\mu_{odd}(\mathcal{B}(V,\pi))-\cancel{\mu_{odd}(\mathcal{B}(W,\theta))}|
\end{eqnarray*}
\end{proof}
\end{frame}


\begin{frame}{4.4 Representaciones en m\'odulos de persistencia}
\framesubtitle{4.4.1 Desarrollo te\'orico}
Recordemos que la representaci\'on de un grupo $G$ es un par $(V,\rho)$, donde $V$ es un espacio vectorial de dimensi\'on finita y $\rho$ es un homomorfismo de $G$ a $\mbox{GL}(V)$. Adaptaremos este concepto a los m\'odulos de persistencia.\\
\vspace{1em}
\begin{block}{Definici\'on 4.4.2}
Una {\bfseries representaci\'on de persistencia de un grupo} $G$ es un par $\left((V,\pi),\rho\right)$, donde $(V,\pi)$ es un m\'odulo de persistencia y $\rho$ es un homomorfismo de $G$ al grupo de automorfismos de persistencia de $(V,\pi)$. Una {\bfseries subrepresentaci\'on de persistencia} $\left( (W,\pi),\rho\right)$ de $\left((V,\pi),\rho\right)$ es un subm\'odulo de persistencia $(W,\pi)$ de $(V,\pi)$ tal que $\forall t\in \mathbb{R}$, $W_t$ es invariante bajo $\rho(g)_t$ para cualquier $g\in G$.
\end{block}
\end{frame}

\begin{frame}
\begin{example2}Un \emph{m\'odulo de persistencia con involuci\'on} (pmi), denotado $((V,\pi),A)$, es una representaci\'on de persistencia de $G=\mathbb{Z}_2$. En otras palabras, $A$ es un homomorfismo de $G$ al grupo de persistencia de automorfismos de $(V,\pi)$ tal que para cualquier $t\in \mathbb{R}$, $A_t^2=\mathbb{1}$. 
\end{example2}
\vspace{1em}
\begin{block}{Definici\'on 4.3.3}
Sean $((V,\pi),\rho^V)$ y $((W,\theta),\rho^W)$ dos representaciones de persistencia del grupo $G$. Un {\bfseries morfismo} $G$-{\bfseries persistente} $\mathfrak{f}:((V,\pi),\rho^V)\rightarrow ((W,\theta),\rho^W)$ es una $\mathbb{R}$-familia de morfismos persistentes $G$-equivariantes $f_t:V_t\rightarrow W_t$, $t\in \mathbb{R}$.
\end{block}
\end{frame}


\begin{frame}
Dado un $G$-morfismo persistente $\mathfrak{f}:((V,\pi),\rho^V)\rightarrow ((W,\theta),\rho^W)$, se puede considerar
\begin{gather*}
\mbox{ker}(\mathfrak{f})=\left\{  v\in V_t \vert f_t(v)=0\right\}_{t\in \mathbb{R}}\\
y\\
\mbox{im}(\mathfrak{f})=\left\{ f_t(v)\in W_t\vert v\in V_t\right\}_{t\in \mathbb{R}}\,.
\end{gather*}

\begin{minipage}{0.25\textwidth}

\end{minipage}\hfill\scalebox{0.8}{\begin{minipage}{0.75\textwidth}
\begin{block}{Ejercicio 4.4.4} Demuestre que $(\mbox{ker}(\mathfrak{f}),\rho^V)$ es una subrepresentaci\'on persistente de $((V,\pi),\rho^V)$ y, an\'alogamente, $(\mbox{im}(\mathfrak{f}),\rho^W)$ de $((W,\theta),\rho^W)$ .
\end{block}
\end{minipage}}\\
\begin{block}{Ejemplo 4.4.5}
Sea $((V,\pi),\rho^V)$ una representaci\'on persistente sobre $\mathbb{C}$ del grupo finito $\mathbb{Z}_p = \{0, 1, \cdots, p - 1\}$. Sea $\xi$ la $p$-\'esima ra\'iz de la unidad. Consideremos $(L_\xi)_t=\mbox{ker}(\rho(1)_t-\xi\mathbb{1}_{V_t})$ para todo $t\in \mathbb{R}$. Entonces $((\{ (L_\xi)_t \}_{t\in\mathbb{R}} , \pi), \rho)$ es una subrepresentaci\'on persistente de $((V,\pi),\rho^V)$.
\end{block}
\end{frame}


\begin{frame}
Recordemos que un $\delta$ corrimiento de un m\'odulo de persistencia $(V,\pi)$, denotado $(V[\delta],\pi[\delta])$, se define como $V[\delta]_{t}=V_{t+\delta}$ y $\pi[\delta]_{s,t}=\pi_{s+\delta,t+\delta}$. An\'alogamente, para cualquier morfismo persistence $\mathfrak{f}:(V,\pi)\rightarrow (W,\theta)$ definimos su $\delta$-corrimiento:
\begin{gather*}
\mathfrak{f}[\delta]:(V[\delta],\pi[\delta])\rightarrow(W[\delta],\theta[\delta])\\
(\mathfrak{f}[\delta])_t=f_{t+\delta}\,.
\end{gather*}
Observe que si $((V,\pi),\rho)$ es una representaci\'on persistente de $G$, tambi\'en lo es $((V[\delta],\pi[\delta]),\rho[\delta])$ .
\end{frame}

\begin{frame}{Definici\'on 4.4.7}
Sean $((V,\pi),\rho^V)$ y $((W,\theta),\rho^W)$ dos representaciones de persistencia del grupo $G$. Decimos que $(V,\pi)$ y $(W,\theta)$ est\'an $(\delta,G)$-entrelazados si existen $G$-morfismos persistentes $\mathfrak{f}:(V,\pi)\rightarrow(W[\delta],\theta[\delta])$ y $\mathfrak{g}:(W,\theta)\rightarrow(V[\delta],\pi[\delta])$ tales que los siguientes diagramas conmutan:
\begin{center}
\includegraphics[scale=0.4]{diagrams/fig09.png}
\end{center}
\end{frame}


\begin{frame}
Consistentemente, podemos definir la distancia de $G$-entrelazamiento como
\begin{gather*}
d_{G-int}\left( (V,\pi),\,(W,\theta)\right)=\mbox{inf}\left\{\delta >0\vert (V,\pi) \mbox{ y } (W,\theta) \mbox{ est\'an }\right.\\
\, \hspace{20em} \left. (\delta,G)\mbox{-entrelazados} \right\}\,.
\end{gather*}

\vspace{2em}

\begin{block}{Proposici\'on 4.4.8}
Sean $((V,\pi),\rho^V)$ y $((W,\theta),\rho^W)$ dos representaciones de persistencia del grupo $G$. Entonces,
\begin{gather*}
d_{G-int}\left( (V,\pi),\,(W,\theta)\right)\geq d_{int}\left( (V,\pi),\,(W,\theta)\right)\,.
\end{gather*}
\end{block}
\end{frame}

\begin{frame}
\begin{block}{Ejemplo 4.4.9}
Sean $((V,\pi),\rho^V)$ y $((W,\theta),\rho^W)$ dos representaciones de persistencia sobre $\mathbb{C}$ del grupo finito $\mathbb{Z}_{p}=\{0,1,\cdots,p-1\}$, con $p$ un n\'umero primo. Para cualquier ra\'iz $p$-\'esima de la unidad $\xi$, denotemos como $((L_{\xi}^{V},\pi),\rho^V)$ y $((L_{\xi}^{W},\theta),\rho^W)$ a las subrepresentaciones de persistencia de $((V,\pi),\rho^V)$ y $((W,\theta),\rho^W)$, respectivamente (ver Ejemplo 4.4.5). Si $((V,\pi),\rho^V)$ y $((W,\theta),\rho^W)$ est\'an $(\delta,G)$-entrelazados, se puede demostrar que $((L_{\xi}^{V},\pi),\rho^V)$ y $((L_{\xi}^{W},\theta),\rho^W)$ tambi\'en est\'an $(\delta,G)$-entrelazados. Por tanto,
\begin{eqnarray*}
d_{G-int}((V,\pi),(W,\theta))&\geq& d_{G-int}\left( (L_{\xi}^{V},\pi), \,(L_{\xi}^{W},\theta)\right)\\
&\geq& d_{int}\left( (L_{\xi}^{V},\pi), \,(L_{\xi}^{W},\theta)\right)\\
&=&d_{bot}\left( (L_{\xi}^{V},\pi), \,(L_{\xi}^{W},\theta)\right)\,.
\end{eqnarray*}
\end{block}
\end{frame}

\begin{frame}
En esta secci\'on no se trabaja con el caso general $G=\mathbb{Z}_{p}$ sino s\'olo con $p=2,4$. Note que si $\mathbb{Z}_4$ act\'ua en un conjunto, esta acci\'on induce una $\mathbb{Z}_2$-acci\'on sobre el mismo conjunto, por el homomorfismo $\mathbb{Z}_2\rightarrow\mathbb{Z}_4$, $1\mapsto 2$. Decimos que un módulo de persistencia con involución (pmi) $((W,\theta),B)$ es un $\mathbb{Z}_4$-pmi si su $\mathbb{Z}_2$ acci\'on $B$ proviene de una $\mathbb{Z}_4$-acci\'on, es decir, si existe un morfismo de persistencia $C:(W,\theta)\rightarrow (W,\theta)$, tal que $B=C^2$ y $C^4=\mathbb{1}$.
\vspace{2em}

Sea $((V,\pi),A)$ un pmi. Retomando el Ejemplo 4.4.5 con $\xi=-1$, denotemos $L^V$ al m\'odulo de persistencia resultante construido a partir de los $(-1)-$espacios propios.
\end{frame}


\begin{frame}{Teorema 4.4.11}
Sea $((V,\pi),A)$ un módulo de persistencia con involución (pmi). La $\mathbb{Z}_2$-distancia de entrelazamiento entre $V$ y la collecci\'on de m\'odulos de persistencia con involuci\'on cuya $\mathbb{Z}_2$ acci\'on proviene de una $\mathbb{Z}_4$, est\'a acotada por abajo en t\'erminos de la funci\'on de multiplicidad:\\
Para cualquier $\mathbb{Z}_4$-pmi $((W,\theta),B)$,
\begin{gather*}
d_{\mathbb{Z}_2-int}\left(V,W\right)\geq \mu_{odd}\left(L^V \right)\,.
\end{gather*}
\begin{proof}
\begin{enumerate}
\item[] Ejercicio 4.4.12:
\item Demostrar que si $(V,\pi)$ y $(W,\theta)$ est\'an $\mathbb{Z}_2$-entrelazados, entonces $L^V$ y $L^W$ est\'an $\delta$-entrelazados.
\item Demostrar que $C(L^W)=L^W$, y deducir que $C^2\vert_{L^W}=-\mathbb{1}$ .
\end{enumerate}
En consecuencia, $L^W$ es un m\'odulo de peristencia complejo (Definici\'on 4.3.3). Por tanto, por la Afirmaci\'on 4.3.5,
\vspace{-0.7em}
\begin{gather*}
d_{\mathbb{Z}_2-int}\left((V,\pi),(W,\theta)\right)\geq d_{int}(L^V,L^W)\geq \mu_{odd}\left(L^V \right)\,.
\end{gather*}
\end{proof}
\end{frame}
\end{document}




