\documentclass{beamer}
\setbeamertemplate{theorems}[numbered]
\usecolortheme{dracula}
\usepackage[utf8]{inputenc}
\usepackage[
  main=spanish
]{babel}

\usepackage{amsmath,amsthm,amsfonts,amssymb}
\usepackage{bm}
\usepackage{graphicx}

\newcounter{Ejercicio}
%\newcounter{Ejemplo}

%\newtheorem{Theorem}{Teorema}[section]
%\newtheorem{Lemma}[Theorem]{Lema}
%\newtheorem{Corollary}[Theorem]{Corolario}
%\newtheorem{Ejercicio}[Theorem]{Ejercicio}
%
%
\newtheorem{Ejercicio}[theorem]{Ejercicio}%[count-ejercicio]

\newtheorem{Ejemplo}{Ejemplo}

%\newtheorem{Proposition}[Theorem]{Proposici\'on}
%\newtheorem{Conjecture}[Theorem]{Conjecture}
%\newtheorem{Definition}[Theorem]{Definici\'on}
%\newtheorem{Example}[Theorem]{Ejemplo}
%\newtheorem{Observation}[Theorem]{Observation}
%\newtheorem{Remark}[Theorem]{Remark}

\def \rk{{\mbox {rk}}\,}
\def \dim{{\mbox {dim}}\,}
\def \ex{\mbox{\rm ex}}
\def\df{\buildrel \rm def \over =}
\def\ind{{\mbox {ind}}\,}
\def\Vol{\mbox{Vol}}
\def\V{\mbox{Var}}
\newcommand{\comp}{\mbox{\tiny{o}}}
\newcommand{\QED}{{\hfill$\Box$\medskip}}
\newcommand\norm[1]{\lVert#1\rVert}
\newcommand{\calB}{{\mathcal{B}}}
\newcommand{\calC}{{\mathcal{C}}}
\DeclareMathOperator{\coim}{coim}
\DeclareMathOperator{\im}{im}
\newcommand{\restr}{\big|}  % restricted to %


\def\dint{\operatorname{d}_{\operatorname{int}}}
\def\dbot{\operatorname{d}_{\operatorname{bot}}}
\def\Z{{\bf Z}}
\def\R\re
\def\V{\bf V}
\def\W{\bf W}
\def\f{\tilde{f}_{k}}
\def \e{\varepsilon}
\def \la{\lambda}
\def \vr{\varphi}
\def \R{{\bf R}}
\def \L{{\mathcal L}}

\def \FF{{\mathbb F}}
\def \re{{\mathbb R}}
\def \Q{{\mathbb Q}}
\def \cp{{\mathbb CP}}
\def \T{{\mathbb T}}
\def \C{{\bf C}}
\def \M{{\widetilde{M}}}
\def \I{{\mathbb I}}
\def \H{{\mathbb H}}
\def \lv{\left\vert}
\def \rv{\right\vert}
\def \ov{\overline}
\def \tx{{\widehat{x}}}
\def \0{\lambda_{0}}
\def \la{\lambda}
\def \ga{\gamma}
\def \de{\delta}
\def \x{\widetilde{x}}
\def \E{\mathbb{E}}
\def \y{\widetilde{y}}
\def \A{{\mathcal A}}
\def\h{{\rm h}_{\rm top}(g)}
\def\en{{\rm h}_{\rm top}}
\def\F{{\mathcal F}}
\def\co{\colon\thinspace}

\usepackage{ragged2e}  % `\justifying` text
\usepackage{booktabs}  % Tables
\usepackage{tabularx}
\usepackage{tikz}      % Diagrams
\usetikzlibrary{calc, shapes, backgrounds}
\usepackage{amsmath, amssymb}
\usepackage{url}       % `\url`s
\usepackage{listings}  % Code listings
\usepackage{dsfont}
\usepackage{mathtools}
\usepackage{stmaryrd}
\usepackage{bbold}
\usepackage{xfrac}
\usepackage{tikz-cd}


\title{Seminario Persistencia en Geometría 2024-I}
\subtitle{Secciones 3.2.4-3.2.11} 
\author{Miguel Ángel Maurin García de la Vega}
\logo{
\includegraphics[width=1cm]{LogoIMUNAM_Bco.png}
}

\begin{document}

\frenchspacing

\setbeamertemplate{caption}{\raggedright\insertcaption\par}

  \frame{\maketitle}

  \AtBeginSection[]{% Print an outline at the beginning of sections
    \begin{frame}<beamer>
      \frametitle{Contenidos}
      \tableofcontents[currentsection]
    \end{frame}}

    \section{Recapitulación}

    \begin{frame}{Teorema de Isometría}

En el espacio de módulos de persistencia definimos la distancia de $\delta$-entrelazamiento.
\pause
Luego, el Teorema de la Forma Normal (2.1.2) nos permitió asociar a cualquier módulo de persistencia un código de barras. 
\pause
En el espacio de códigos de barras definimos la distancia cuello de botella.
\pause
El teorema de la Isometría establece una relación entre todos estos conceptos:
 
\begin{block}{Teorema 2.2.8:}
El mapa \(V \mapsto \mathcal{B}(V)\) es una isometría:

\[\dint(V,W) = \dbot(\mathcal{B}(V),\mathcal{B}(W))\]
\end{block}
\end{frame}
    
\begin{frame}{Estrategia Demostración Isometría}

El objetivo de la Sección 3 es demostrar este teorema. Al tratarse de una igualdad, hay que demostrar dos desigualdades: 
\pause

\begin{enumerate}
  \item \(\dint(V,W) \leq \dbot(\mathcal{B}(V),\mathcal{B}(W))\)
  \pause
  \item \(\dint(V,W) \geq \dbot(\mathcal{B}(V),\mathcal{B}(W))\)
\end{enumerate}
\pause
La primera desigualdad (Teorema 3.1.1) es consecuencia del Teorema de la forma Normal y del hecho que si dos intervalos están $\delta$-emparejados, entonces, sus módulos de persistencia están $\delta$-entrelazados. (Ejercicio 2.2.7)
\pause
Queremos probar la segunda desigualdad (Teorema 3.1.2) llamada también Teorema de Estabilidad Algebraica.
\end{frame}


\begin{frame}{Estrategia Demostración Estabilidad Algebraica}
De manera análoga a la primera demostración, queremos probar que dados dos módulos de persistencia $\delta$-entrelazados, es posible construir un $\delta$-emparejamiento entre sus códigos de barras.
\pause
Haremos esto en dos pasos: Primero, hay que encontrar una manera de asociar un emparejamiento a cualquier morfismo entre módulos de persistencia. Es decir, inducir emparejamientos. 
\pause
Segundo, veremos que tal construcción al ser aplicada a morfismos de $\delta$-entrelazamiento produce $\delta$-emparejamientos.\\ 
\pause
El resto de esta presentación está centrado en la construcción del primer paso. 
\end{frame}

\section{3.2.2 Construcción de Emparejamientos Inducidos}
\begin{frame}{Proposición 3.2.1}

En lo siguiente, consideramos \((V,\pi)\) y \((W,\theta)\) módulos de persistencia, con correspondientes códigos de barras \(\mathcal{B}\) y \(\mathcal{C}\).
\pause

Antes de empezar la construcción, recordemos un resultado que obtuvimos anteriormente. 
\pause

Si \(\mathcal{B}\) es un código de barras e \(I=(b,d]\) un intervalo, denotamos por \(\mathcal{B}^-_I\subset\mathcal{B}\) a la colección de barras \((a,d]\in\mathcal{B}\) con \(a\leq b\). Así:

\begin{block}{Proposición 3.2.1:}
Si \(\iota:(V,\pi)\to(W,\theta)\) es injectivo, entonces, \(\#\mathcal{B}^-_I\leq\#\mathcal{C}^-_I\)

\end{block}

\end{frame}

\begin{frame}{Emparejamiento para morfismos inyectivos}

Primero, construiremos emparejamientos inducidos por morfismos para los casos particulares de morfismos inyectivos y suprayectivos. Después, combinaremos estos resultados para morfismos en general.
\pause
Para el caso de morfismos inyectivos tenemos:
\pause

\begin{block}{Definición 3.2.4}
Supongamos que $\iota: V \to W$ es \textbf{inyectivo}.
	Definimos el \emph{emparejamiento inducido} $\mu_{inj} : \calB \to \calC$ de la siguiente manera: Para cada $d\in \R \cup \{\infty\}$, ordenamos la barras de $\calB$ de la forma $(\cdot, d]$ siguiendo el orden de ``más largas primero":
	$$
	(b_1, d] \supset (b_2, d] \supset \ldots \supset (b_k, d] \ \text{ en $\calB$, con } b_1\leq b_2 \leq \ldots \leq b_k \;,
	$$
 \pause
	de manera similar para $\calC$:
	$$
	(c_1, d] \supset (c_2, d] \supset \ldots \supset (c_l, d] \ \text{ en $\calC$, con } c_1\leq c_2 \leq \ldots \leq c_l \;.
	$$	
 \pause
 Notamos que por (3.2.1) debe ser $k\leq l$.
 \end{block}
 \end{frame}
	\begin{frame}{Emparejamiento para morfismos inyectivos}
	    
	\begin{block}{Cont. Def. 3.2.4}
	Ahora, emparejamos las barras siguiendo el orden de ``más largas primero". Es decir, en cada paso, tomamos el intervalo más largo de la primera lista y lo emparejamos con el más largo de la segunda.
 \pause
	Repetimos para toda $d \in \R \cup \{\infty\}$ y obtenemos un emparejamiento $\mu_{inj} : \calB \to \calC$.
\end{block}
\end{frame}

\begin{frame}{Emparejamiento para morfismos inyectivos}
\begin{block}{Proposición 3.2.5}
		Si hay un morfismo inyectivo de $(V, \pi)$ a $(W, \theta)$, entonces, el emparejamiento inducido $\mu_{inj}: \calB \to \calC$ cumple:
	\begin{enumerate}[(1)]
		\item
			$\coim \mu_{inj} = \calB$,
   \pause
		\item
			Para todo $(b,d] \in \calB$, $\mu_{inj} (b,d] = (c,d]$ con $c\leq b$.
	\end{enumerate}	
\end{block}
\pause
\begin{block}{Observación 3.2.6}
    El emparejamiento inducido no depende del morfismo $\iota$, solo del supuesto de existencia de un morfismo inyectivo. 
\end{block}
\end{frame}

\begin{frame}{Emparejamiento para morfismos inyectivos}
\begin{block}{Demostración}
Ambas propiedades son consecuencia de la proposición 3.2.1
\pause
\begin{enumerate}[(1)]
		\item
			Como emparejamos ``más largas primero", tenemos que $\mu (b_i,d] = (c_i, d]$. Para el intervalo $(b_k,d]$, sabemos que $k\leq l$, por lo que todas las barras de $\cal B$ están emparejadas.
   \pause
		\item
			 Para los intervalos $(b_i, d]$, aplicando inductivamente la proposición, tenemos que $b_i \leq c_i$ para cada $1\leq i\leq k$.
	\end{enumerate}
		
\end{block}
\end{frame}
\begin{frame}{Emparejamiento para morfismos suprayectivos}

Ahora, consideraremos el caso de un morfismo suprayectivo. Tenemos una construcción y propiedades análogas.
\pause


\begin{block}{Definición 3.2.7}
Supongamos que $\sigma: V \to W$ es \textbf{suprayectivo}.
	Definimos el \emph{emparejamiento inducido} $\mu_{sur}: \calB \to \calC$ como sigue:
	Para cada $b\in \R$, ordenamos los intervalos $(b, \cdot] \in \calB$ de manera decreciente:
	$$
	(b, d_1] \supset (b, d_2] \supset \ldots \supset (b, d_k] \ \text{ en $\calB$, con } d_1\geq d_2 \geq \ldots \geq d_k \;,
	$$
	
 \pause
	de manera similar para $\calC$:
	$$
	(b, e_1] \supset (b, e_2] \supset \ldots \supset (b, e_l] \ \text{ en $\calC$, con } e_1\geq e_2 \geq \ldots \geq e_l \;.
	$$
\pause
Los emparejamos siguiendo el principio de ``más largas primero'' y ensamblamos estos emparejamientos para toda $b$.

 \end{block}
 \end{frame}


\begin{frame}{Emparejamiento para morfismos suprayectivos}
\begin{block}{Proposición 3.2.8}
		Si hay un morfismo suprayectivo de $(V, \pi)$ a $(W, \theta)$, entonces, el emparejamiento inducido $\mu_{sur}: \calB \to \calC$ cumple:
	\begin{enumerate}[(1)]
		\item
			$\im \mu_{sur} = \calC$,
   \pause
		\item
			$\mu_{inj} (b,d] = (b,e]$ con $d\geq e$.
	\end{enumerate}	
\end{block}
\pause
La demostración es análoga a la anterior considerando ahora la proposición 3.2.3 (\(\#\mathcal{B}^+_I\geq\#\mathcal{C}^+_I\))
\end{frame}

\begin{frame}{Emparejamiento para morfismo general}
Para cualquier morfismo \(f: V \to W\) podemos considerar la siguiente descomposición: 
\begin{center}
\includegraphics[scale=0.65]{index.png}
\end{center}
\pause
Usando las proposiciones 3.2.8 y 3.2.5 al tener estos morfismos podemos construir los emparejamientos inducidos $\mu_{sur}: \calB(V) \to \calB(\im f)$ y
	$\mu_{inj}: \calB(\im f)\to \calB(W)$. Los cuales nos permiten dar la siguiente definición:
 \pause
 \begin{block}{Definición 3.2.9}
Para un morfismo general $f$, definimos el \emph{emparejamiento inducido} $\mu(f) : \calB (V) \to \calB (W)$ como la composición
	$\mu(f) = \mu_{inj} \circ \mu_{sur}$,
 la cual esta definida ya que $\im \mu_{sur} = \calB (\im f) = \coim (\mu_{inj})$.
 \end{block}

\end{frame}



\section{Ejemplo 3.2.11}
\begin{frame}{Ejemplo 3.2.11}

Consideramos $V = \F (1,3] \oplus \F (1,2]$ y $W = \F (3,4] \oplus \F (0,2]$, sea $f: V \to W$ un morfismo definido como
	$f\restr_{\F(1,3]} = 0$ y
	$f \restr_{\F(1,2]} : \F(1,2] \to \F(0,2]$ (multiplicación por 1)
	
	Entonces, $\im f = 0 \oplus \F(1,2] \subset \F(3,4] \oplus \F(0,2]$. 
 \pause
 \begin{center}
\includegraphics[scale=0.35]{ej3211.png}
\end{center}

\end{frame}

\begin{frame}{Ejemplo 3.2.11}

Así, $\mu_{sur} : (1,3] \mapsto (1,2]$, $\mu_{inj} (1,2] \mapsto (0,2]$.
	Entonces, el emparejamiento inducido $\mu(f): \calB(V) \to \calB(W)$ lleva a $\mu (f) : (1,3] \mapsto (0,2]$, apesar que $f \restr_{\F(1,3]} = 0$.
\end{frame}



\end{document}

